\chapter{Teoria elementare dei numeri}
\label{lezione1}
\epigraph{\textit{\enquote{La matematica è la regina delle scienze e la teoria dei numeri è la regina della matematica. \\ Non è la conoscenza ma l'atto di imparare, non il possesso ma l'arrivarci, che danno la gioia maggiore.}}}{Carl Friedrich Gauss}
Viene detta \enquote{elementare} non per la semplicità degli argomenti che tratta, quanto invece per la relativa semplicità degli strumenti matematici che utilizza. \\ \\
Per ora lavoreremo solo sul ben noto \textit{insieme dei numeri interi}
\begin{equation*}
	\mathbb{Z}= \left\{\dots,-2,-1,0,1,2,\dots\right\}
\end{equation*}
ma in seguito estenderemo i risultati ai più generali \textit{domini d'integrità}. \\ 
Vediamo adesso alcuni brevi richiami di algebra; dovrebbero essere tutti fatti noti, ma saranno fondamentali. Per saperne di più sulla storia della teoria dei numeri, si veda \cite{D19}\cite{G97}.


\section{Aritmetica e numeri interi}
\subsection{$\mathbb{Z}$: elementi riducibili o primi?}
\begin{definizione}[Divisore] Siano $a,b \in \mathbb{Z}$; diciamo che \textbf{$a$ divide $b$} (oppure che \textbf{$a$ è un divisore di $b$}) se 
	\begin{equation*}
	\exists \ x \in \mathbb{Z} \ \text{tale che} \ b = a \cdot c
	\end{equation*}
	ed in notazione lo scriveremo come $a \mid  b$.
\end{definizione}
\begin{proposizione}
	Siano $a,b$ in $\mathbb{Z}$. Allora valgono i seguenti:
	\begin{enumerate}
		\item $c\mid a, \ c\mid b \ \implies \ c\mid (a+b)$
		\item $c\mid a, \ c\mid (a+b) \ \implies \ c\mid b$
	\end{enumerate}
\end{proposizione}
\begin{proof}\ 
\begin{enumerate}
	\item Se $c\mid a$ e $c\mid b$ allora $a=k_a c$ e $b = k_b c$, otteniamo subito che $(a+b)=c(k_a+k_b)$. Concludiamo notando che $k_a+k_b$ appartiene a $\mathbb{Z}$.
	\item Come sopra, $a=k_ac$ ed $(a+b)=kc$ quindi $b=(k-k_a)c$. \\ Abbiamo trovato un elemento di $\mathbb{Z}$.
\end{enumerate}
\end{proof}
\begin{definizione}
	Sia $m$ appartenente a $\mathbb{Z}$, $m \neq \pm 1, 0$.
	\begin{enumerate}
		\item $m$ si dice \textbf{irriducibile} se per ogni $a$ e $b$ in $\mathbb{Z}$ con $m=ab$ allora uno tra $a$ e $b$ è invertibile (rispetto al prodotto).
		\item $m$ si dice \textbf{riducibile} se non è irriducibile (banalmente: se esistono degli $a$, $b$ in $\mathbb{Z}$ con $m=ab$ e nessuno dei due è invertibile rispetto al prodotto)
		\item $m$ si dice \textbf{primo} se per ogni $a$ e $b$ in $\mathbb{Z}$ con $m\mid ab$, allora $m\mid a$ oppure $m\mid b$. \footnote{Euclide postula che \enquote{\textit{un numero è primo se è misurato dall'unità}}}.
	\end{enumerate}
\end{definizione}
Intuitivamente i \textit{riducibili} sono detti tali perché possono essere \enquote{ridotti} ad un prodotto \enquote{sensato}. \\ \\ Le definizioni di irriducibili e primi in $\mathbb{Z}$ potrebbero trarci in inganno, perché la definizione a cui casualmente ci riconduciamo per i primi è in realtà la definizione formale degli irriducibili; in effetti è curioso, ed è dovuto al fatto che in $\mathbb{Z}$ le due definizioni coincidano (lo dimostreremo). \\ Altro fatto interessante è che accetteremo i numeri negativi come numeri primi.
\begin{esempio} Vediamo alcuni facili esempi.
	\begin{itemize}
		\item[$(8)$] Si noti che $8\mid 40$ e $40=20\cdot 2$, inoltre $8$ non divide né $20$ né $2$. \\
		Allora 8 non è primo.
		\item[$(-7)$] Siccome $-7 = (-7)(1)=(7)(-1)$, allora è irriducibile.
		\item[$(12)$] Siccome $12 = (3)(4)$, allora è riducibile. 
		\item[$(5)$] Siccome $5 = (5)(1)=(-5)(-1)$, allora è irriducibile. 
		\item[$(-6)$] Siccome $-6 = (-3)(2)$, allora è riducibile. 
	\end{itemize}
\end{esempio}
\begin{definizione}[$\MCD$] Siano $a$ e $b$ in $\mathbb{Z}$ non entrambi nulli, allora diciamo loro \textbf{massimo comune divisore}:
	\begin{equation*}
		\MCD(a,b)=\max\left\{d \in \mathbb{Z} \ \text{tali che} \ d\mid a, \ d\mid b\right\}
	\end{equation*}
\end{definizione}
\begin{osservazione}
	Notiamo che il $\MCD$ è sempre positivo, infatti se uno dei due fosse zero varrebbe che \begin{equation*}
	\MCD(a,b)= |b|
	\end{equation*}
\end{osservazione}
\begin{teorema}[Divisione euclidea] Siano $a$ e $b$ in $\mathbb{Z}$, con $b \neq 0$. Allora esistono unici $q,r \in \mathbb{Z}$ tali che 
	\begin{equation*}
	a = qb + r \ \text{e vale che} \ 0 \leq r \leq |b|
	\end{equation*}
\end{teorema}
\begin{proof}
	Dimostriamo che $q$ ed $r$ esistono. \\ Supponiamo $a \leq 0$ e $b>0$ senza perdita di generalità; se $a=0$ basta prendere $q=r=0$. Possiamo quindi fissare $b$ e procedere per induzione $a$, con l'ipotesi induttiva che esistano per ogni coppia $(n,b)$ con $n<a$, voglio mostrare che esistono anche per la coppia $(a,b)$.
	\begin{itemize}
		\item Se $a<b$ basta prendere $q=0$ ed $r=q$, senza necessità di induzione.
		\item Se $a\geq b$ allora $a-b=c$ per un certo $c\geq 0$; posso usare l'ipotesi induttiva su $c$, quindi esistono $q'$ ed $r'$ tali che 
		\begin{equation*}
		c = q'b + r' \ \text{e vale che} \ 0 \leq r' \leq |b|
		\end{equation*}
		e chiaramente ottengo subito
		\begin{equation*}
		a = b+c = (q'+1)b + r' \ \text{e vale che} \ 0 \leq r' \leq |b|
		\end{equation*}
	\end{itemize}
Dimostriamo ora l'unicità. \\
Siano $q$ e $q'$, $r$ ed $r'$ tali che 
\begin{equation*}
a = qb + r= q'b + r'\ \text{e vale che} \ 0 \leq r,r' \leq |b|
\end{equation*}
Sia senza perdita di generalità $r \geq r'$, ma allora
\begin{equation*}
(q'-q)b=r-r' \ \ \ \ 0 \leq r-r' \leq |b|
\end{equation*}
Questo vuol dire che $b$ divide $r-r'$, ma per ipotesi $b\geq r,r'$,
quindi necessariamente $r-r'=0$ ed $r=r'$. Immediatamente segue $q=q'$. \\ \\ Notiamo che se non fosse per la condizione a lato l'unicità non varrebbe! \\ Ad esempio 
\begin{equation*}
16 = 4\cdot3+2
\end{equation*}
ma siccome abbiamo eliminato la richiesta $0\leq r \leq |b|=3$ allora posso accettare scritture del tipo
\begin{equation*}
16=10\cdot3-14
\end{equation*}
\end{proof}
\begin{osservazione}[Algoritmo di Euclide] Possiamo costruire un algoritmo per il calcolo del massimo comune divisore di due $a$ e $b$ generici in $\mathbb{Z}$. Ma come? La dimostrazione che abbiamo visto non è costruttiva. \\ \\ Facciamo alcune osservazioni.
	\begin{enumerate}
		\item Sia $c \in \mathbb{Z}$ tale che divide sia $a$ che $b$, allora scriviamo $a=kc$ e $b=hc$. Si noti che
		\begin{equation*}
		a=qb+r \implies kc = qhc+r \implies r = (k-qh)c
		\end{equation*}
		ovvero $c\mid r$, quindi i divisori comuni tra $a$ e $b$ dividono anche $r$. \\ Vediamo un esempio numerico, ovviamente $c=2$ divide sia $a=10$ che $b=8$; $a=2\cdot5$, $b=2\cdot4$, allora
		\begin{equation*}
		10=q8+r\implies(2\cdot5)=q(2\cdot4)+r\implies r=(5-4q)2
		\end{equation*}
		\item Sia ora $c\in \mathbb{Z}$ tale che divide sia $a$ che $r$, ovvero $b=ic$ ed $r=jc$.
		\begin{equation*}
		a=qb+r \implies a = qic+jc =(qi+j)c \implies c\mid a
		\end{equation*}
		quindi i divisori comuni tra $b$ ed $r$ dividono anche $a$.
		\item Unendo quanto appena visto e definendo
		\begin{equation*}
			A = \left\{ d \in \mathbb{Z} \,\middle|\, d \mid a \land d \mid b \right\}
		\end{equation*}
		\begin{equation*}
			B = \left\{ d \in \mathbb{Z} \,\middle|\, d \mid b \land d \mid r \right\}
		\end{equation*}
		possiamo notare che $A$=$B$, per definizione allora vale la seguente:
		\begin{equation*}
			\MCD(a,b)=\MCD(b,r)
		\end{equation*}
	\end{enumerate}
Vediamo \textbf{l'algoritmo di Euclide} per il calcolo di $\MCD(a,b)$.
\begin{equation*}
	\arraycolsep=1.4pt
	\begin{array}{rllll} 
		 

		a 		& = q_1b + r_1 & \quad\ 0 \leq r_1 \leq  |b|\\
		b 		& = q_2r_1 + r_2 & \quad\ 0 \leq r_2 \leq |r_1|\\
		   		& \,\vdots 		  & \\		
		r_{n-2} & = q_{n}r_{n-1}+r_n & \quad\ 0 \leq r_n \leq |r_{n-1}|\\
		r_{n-1} & = q_{n+1}r_n		 & 
	\end{array}
\end{equation*}
Vediamo riassunti in breve i passaggi tramite coppie di elementi di $\mathbb{Z}$:
\begin{equation*}
(a,b) \to (b,r_1)\to\dots\to(r_{n-1},r_n)
\end{equation*}
il loro significato è abbastanza ovvio. Siccome la successione dei resti è una successione di interi strettamente decrescenti e positivi è convergente, ed abbastanza evidentemente converge a $0$. L'ultimo resto non nullo è il massimo comune divisore cercato, ed il resto nullo esiste necessariamente per quanto appena detto: so che è possibile arrivarci con una quantità finita di passi. \\ \\ Ma il valore che otteniamo è \textit{veramente} il massimo comune divisore? Per quanto osservato all'inizio sì, infatti 
\begin{equation*}	
	\MCD(a,b)= \MCD(b,r_1)= \ \dots \ = \MCD(r_{n-1},r_n)= \MCD(r_{n},0) = r_n
\end{equation*}
e questo è un risultato tutt'altro che banale. \\ \\
L'algoritmo di Euclide sarà fondamentale in teoria dei numeri, ad esempio nella costruzione delle \textit{frazioni continue}.
\end{osservazione}
\begin{esempio}
	Proviamo a calcolare il massimo comune divisore tra $76$ e $58$.
	\begin{align*}
		76 &= q_158 + r_1 \ \ \ \ \ \ 0 \leq r_1 \leq |58| \ \implies \ (q_1,r_1)=(1,18)\\
		58 &= q_218 + r_2 \ \ \ \ \ \ 0 \leq r_2 \leq |18| \ \implies \ (q_2,r_2)=(3,4)\\
		18 &= q_34 + r_3 \ \ \ \ \ \ \ \ 0 \leq r_3 \leq |4| \ \ \implies \ (q_3,r_3)=(4,2)\\
		4 &= q_42 + r_4 \ \ \ \ \ \ \ \ 0 \leq r_4 \leq |2| \ \ \implies \ (q_4,r_4)=(2,0)
	\end{align*}
	Allora il massimo comune divisore tra $76$ e $58$ è il primo $r_i$ che precede il calore nullo, nel nostro caso $2$. Il nostro risultato ha senso? Sì, basta scomporre:
	\begin{align*}
	76 &=(19)(2)(2)\\
	58&=(29)(2)
	\end{align*}
	Non potremo sempre permetterci di scomporre, ad esempio non con numeri grandi e brutti, ma perché non provarlo almeno una volta? \\ \\ Cosa succederebbe se invece di scegliere $a=76$ e $b=58$ li invertissimo? Intuitivamente è chiaro che il massimo comune divisore debba risultare identico. Vediamolo.
	\begin{equation*}
	58 = q_176 + r_1 \ \ \ \ \ \ 0 \leq r_1 \leq |76| \ \implies \ (q_1,r_1)=(0,58)
	\end{equation*}
	Abbastanza curiosamente mi riconduco subito al caso iniziale, ma questo era prevedibile. Per risparmiare tempo e calcoli è sempre più comodo prendere $a\geq b$.
\end{esempio}



\subsection{$\mathbb{Z}$: elementi riducibili e primi!}
\begin{teorema}[Identità di Bezout] Siano $a$ e $b$ in $\mathbb{Z}$, $d$ il loro massimo comune divisore. Allora esistono degli $x,y \in \mathbb{Z}$ tali che
	\begin{equation*}
	d=ax+by
	\end{equation*}
\end{teorema}
\begin{proof}
	Sfrutto la sequenza dell'algoritmo di Euclide prendendo $d=r_n$. Ora posso \enquote{scalare} le uguaglianze, ed ottengo
	\begin{align*}
	d &  = r_n = r_{n-2}-q_nr_{n-1}= r_{n-2}-q_n(r_{n-3}-q_{n-1}r_{n-2}) \\ 
	& = (1+q_nq_{n-1})r_{n-2}+(-q_n)r_{n-3} = \dots
	\end{align*}
	Procedendo iterativamente trovo $x$ ed $y$. \\ \\ Questa dimostrazione al contrario di quella della divisione intera con resto è costruttiva, fornisce un algoritmo per ricavare $x$ ed $y$. \\ Sarebbe necessario scrivere esplicitamente $d$ per completare formalmente la dimostrazione, ma la scrittura diventerebbe pesante.
\end{proof}
Posso avere però lo stesso risultato tramite una formalizzazione più algebrica del precedente approccio aritmetico.
\begin{teorema}[Identità di Bezout] \
	\begin{enumerate}
		\item Siano $a$ e $b$ in $\mathbb{Z}$ non entrambi nulli, allora 
		\begin{equation*}
		I \eqqcolon \left\{na+mb \, \middle| \, n,m \in \mathbb{Z}\right\}
		\end{equation*}
		è un ideale di $\mathbb{Z}$.
		\item Sia $I = (d)$, allora $d$ è il massimo comune divisore tra $a$ e $b$. \footnote{in algebra commutativa $I=(d)$ indica che \textbf{$d$ genera l'ideale $I$}, ovvero che ogni $i \in I$ ha forma $kd$ per un $k$ in $\mathbb{Z}$. Stiamo usando il fatto che $\mathbb{Z}$ sia un dominio ad ideali principali.}
	\end{enumerate}
\end{teorema}
\begin{proof}\
	\begin{enumerate}
		\item Perché $I$ sia un ideale, prima di tutto $(I,+)$ deve essere un sottogruppo di $(\mathbb{Z},+)$, il che è banalmente ovvio. Inoltre la differenza di due elementi $x,y \in I$ deve appartenere ad $I$. Ma questo è ovvio per costruzione,
		\begin{align*}
		x &= na+mb \\y&=ia+jb
		\end{align*}
		\begin{equation*}
		\implies x-y = (n-i)a+(m-j)b \in I
		\end{equation*}
		Altra condizione è che per ogni elemento $k$ dell'anello e per ogni $\alpha$ di $I$, $k\alpha$ appartenga ad $I$.
		\begin{equation*}
		k\alpha = k(na+mb)=(kn)a+(km)b\in I
		\end{equation*}
		\item Se $I=(d)$ ho che
		\begin{align*}
		a&=1a+0b \in I\\ b&=0a+1b \in I
		\end{align*}
		e quindi per ipotesi su $I$ ho che $a=cd$, $b=kd$. Inoltre se $d \in I$ allora posso usare la prima identità:
		\begin{equation*}
		\exists \ x,y \in \mathbb{Z} \ |\ d=xa+yb
		\end{equation*}
		Da questa ottengo che se esiste un $w$ che divide sia $a$ che $b$, allora $w$ divide $xa$ ed $yb$, quindi divide $d$. 
	\end{enumerate}
\end{proof}
\begin{esempio}
	Utilizzare l'algoritmo di Eulero per il massimo comune divisore di $3522$ e $321$, esprimerlo con l'identità di Bezout.
	\begin{equation*}
		\arraycolsep=1.4pt
		\begin{array}{rl}
			3522 &= (10)(321)+312\\
			321 &= (1)(312)+9\\
			312 &= (34)(9)+6\\
			9 &=(1)(6)+3\\
			6 &=(2)(3)+0
		\end{array}
	\end{equation*}
	Abbiamo già visto la procedura, quindi non ci soffermeremo troppo. Per esprimere $d$ come da identità basta ripercorrere la dimostrazione, ovvero 
	\begin{align*}
	\MCD(a,b)=3&= 9-6=9-(312-34\cdot9)=-312+35\cdot9\\
	&= -312+35(321-312)=35\cdot321-36\cdot312\\
	&= 35\cdot321-36\cdot(3522-10\cdot321)=-36\cdot3522+395\cdot321
	\end{align*}
	I calcoli diventano molto brutti molto in fretta.
\end{esempio}
\begin{teorema}
	Sia $n \in \mathbb{Z}$ diverso da $\pm 1,0$, allora
	\begin{equation*}
	n \ \text{irriducibile} \ \iff \ n \ \text{primo}
	\end{equation*}
\end{teorema}
\begin{proof}\
	\begin{itemize}
		\item[$\implies$] Devo dimostrare che $n$ è primo, supponiamo che $n\mid ab$; devo dimostrare che $n\mid a$ oppure $n\mid b$ cerchiamo di dimostrare ad esempio che $n$ non divide $a$ ma divide $b$.
		\begin{equation*}
		n\mid ab \implies ab=kn \ \ \ k \in \mathbb{Z}
		\end{equation*}
		Siccome $n$ è irriducibile, i suoi soli divisori sono $\pm 1$ e $\pm n$, dunque il massimo comune divisore tra $a$ ed $n$ è 1. Per Bezout 
		\begin{equation*}
		1=xa+yn
		\end{equation*}
		\begin{equation*}
		b=b\cdot1=b(xa+yn)=x(ab)+n(by)=(xk+by)n
		\end{equation*}
		ed abbiamo ottenuto che $n$ divide $b$.
		\item[$\impliedby$] Sia $n=ab$, allora $n$ divide $a$ oppure $n$ divide $b$. Supponiamo senza perdita di generalità che valga solo la prima, allora $a=kn$ per un certo $k$ intero.
		\begin{equation*}
		n = ab = nkb \implies 1 = kb \implies b = \pm 1
		\end{equation*}
		ed abbiamo ottenuto che $n$ è irriducibile.
	\end{itemize}
\end{proof}
\begin{teorema}[Teorema fondamentale dell'aritmetica] 
	Sia $n \in \mathbb{Z}$ diverso da $\pm 1,0$; allora esiste una ed un'unica fattorizzazione del tipo
	\begin{equation*}
	n = \pm p_1\cdot p_2 \cdot \ \dots\ \cdot p_s
	\end{equation*}
	composta da $p_i$ primi positivi di $\mathbb{Z}$
\end{teorema}
\begin{proof} Dimostriamo che esiste. \\ Suppongo senza perdita di generalità che $n>0$, se $n=2$ ho già la fattorizzazione
	\begin{equation*}
	n = 2 = p_1 \ \ (s=1)
	\end{equation*}
	Ma allora posso usare un grande classico, l'induzione: procediamo inducendo su $n$. Se la fattorizzazione esiste per ogni intero tra $2$ ed $n-1$ (inclusi):
	\begin{itemize}
		\item Se $n$ è irriducibile allora è primo per caratterizzazione dei primi in $\mathbb{Z}$, e come prima $n=p_1$ con $s=1$.
		\item Se $n$ è riducibile, allora $n=ab$ con $a,b<n$; ma per ipotesi induttiva le fattorizzazioni di $a$ e $b$ in primi esistono, basta moltiplicarle tra di loro.
	\end{itemize}
	Dimostriamo l'unicità. \\ Pongo $n>0$ senza perdita di generalità e suppongo di avere
	\begin{equation*}
	 n= p_1\cdot p_2 \cdot \ \dots\ \cdot p_s= q_1\cdot q_2 \cdot \ \dots\ \cdot q_t \ \ (s \leq t)
	\end{equation*}
	Posso osservare che $p_1$ divide $n$, ed essendo primo si ha che $p_1$ divide un $q_i$ per un qualche indice $i$. Ma siccome $q_i$ è irriducibile deve necessariamente essere $p_1=q_i$. Allora ottengo una seconda fattorizzazione rimuovendoli, ripetendo il ragionamento $s$ volte otterrei (a meno di riordinamento)
	\begin{equation*}
	1 = q_{s+1}\cdot \ \dots \ q_t
	\end{equation*}
	che è un assordo siccome 1 non ha fattori irriducibili: allora deve essere $s=t$. \\ Abbiamo già visto la necessità di $q_j = q_i$ per certi indici $i$ e $j$ opportuni, questo chiude la dimostrazione.
\end{proof}
Più avanti cercheremo di ottenere risultati \enquote{simili} al teorema fondamentale ma su strutture diverse dall'anello $\mathbb{Z}$. Non sempre sarà una situazione facile da trattare come quella dei numeri interi, ad esempio gli invertibili rispetto al prodotto potrebbero non essere $\pm 1$! \\ Avere risultati del genere sarà più fondamentale e meno scontato rispetto agli interi, è importante tenere a mente che ci troviamo in una situazione privilegiata.




\subsection{L'infinità dei numeri primi}
\label{lezione2}
Siccome siamo tra matematici potete ammetterlo tranquillamente, quante volte vi è capitato di avere il seguente discorso?\\
- \enquote{\textit{Quanti sono i numeri primi?}}\\
- \enquote{\textit{Tanti.}}\\ \\
Se la risposta è \textit{nessuna}, siamo ancora in tempo per rimediare.
\begin{teorema}
	Esistono infiniti numeri primi.
\end{teorema}
\begin{proof}[Dimostrazione di Euclide] Supponiamo che i numeri primi siano finiti, allora possiamo prenderli come $p_1, \ \dots, \ p_n$ (positivi). Sia
	\begin{equation*}
	N = p_1 \cdot \ \dots \ \cdot p_n +1
	\end{equation*}
	Siccome ci troviamo in $\mathbb{Z}$ possiamo applicare il teorema fondamentale dell'algebra, esiste una ed un'unica fattorizzazione in primi di $N$. Allora uno dei primi che lo fattorizza è anche un suo divisore, ma necessariamente non è uno dei $p_i$; se lo fosse infatti avremmo 
	\begin{equation*}
	p\mid N, \ \ p\mid p_1 \cdot \ \dots \ \cdot p_n
	\end{equation*}
	e per proprietà dei divisori questo implica che $p\mid 1$, assurdo a causa della definizione di primo.
\end{proof}
\begin{proof}[Dimostrazione di Eulero] 
	Supponiamo che i numeri primi siano finiti, allora possiamo prenderli come $p_1,\dots, p_n$. Per fatti ben noti di analisi matematica possiamo costruire una serie geometrica di natura convergente
	\begin{equation*}
	\sum_{k=0}^{+\infty}\left(\frac{1}{p_i}\right)^k=\frac{1}{1-\left(\frac{1}{p_i}\right)} \ \ \ i=1,\dots,n
	\end{equation*} 
	Allora possiamo riscrivere come:
	\begin{equation*}
	\prod_{i=1}^n\left(\sum_{k=0}^{+\infty}\left(\frac{1}{p_i}\right)^k\right)=\prod_{i=1}^n\left(\frac{1}{1-\left(\frac{1}{p_i}\right)}\right)
	\end{equation*}
	Ma questo è un assurdo. Infatti osserviamo i due membri, a destra abbiamo un valore chiaramente finito ed a sinistra 
	\begin{equation*}
	\left(1+\frac{1}{p_1}+\frac{1}{p_1^2}+ \ \dots\right)\left(1+\frac{1}{p_2}+\frac{1}{p_2^2}+ \ \dots\right)\dots
	\end{equation*}
	che è un valore infinito, perché per teorema fondamentale dell'aritmetica deve darmi la sommatoria $\sum_{n=1}^{+\infty}\frac{1}{n}$. Si noti che sto ancora usando l'ipotesi di finitezza dei primi, ed ho ottenuto la serie armonica (divergente).
\end{proof}
\begin{proof}[Dimostrazione di Métrod] 
	Supponiamo che i numeri primi siano finiti, allora possiamo prenderli come $p_1, \ \dots, \ p_n$. Consideriamo
	\begin{equation*}
	N = p_1 \cdot \ \dots \ \cdot p_n, \ \ \ Q_i = \frac{N}{p_i} \ \ i=1,\dots,n
	\end{equation*}
	allora palesemente $p_i$ non divide $Q_i$ per ogni $i$, ma divide $Q_j$ per ogni $i\neq j$. Sia $S=\sum_{i=1}^nQ_i$, esiste un primo $q$ che divide $S$ (per teorema fondamentale dell'algebra) e non sia nessuno dei $p_i$ (questo per costruzione dei $Q_i$).
\end{proof}
\begin{proof}[Dimostrazione di Cerruti] 
	Consideriamo la \textit{funzione di Legendre}: 
	%todo: funzione di legendre? wtf? introdurre le nuove funzioni?
	% IDEA: Fare un formulario alla fine del libro con cose interessanti di particolari funzioni aritmetiche (?)
	\begin{equation*}
	\phi(x,y)=\left|\left\{n\in \mathbb{N} \, \middle| \, \text{$1\leq n \leq x$, $n$ non ha fattori primi $p\leq y$}  \right\}\right|
	\end{equation*}
	questa funzione è definita per ogni $x,y \in \mathbb{N}^*$.\footnote{$\mathbb{N}^*$ è l'insieme dei naturali meno lo zero, questa notazione si usa spesso in teoria dei gruppi per definire gruppi abeliani moltiplicativi del tipo $(\mathbb{N}^*, \cdot)$.}
	\\ Consideriamo inoltre la \textit{prime counting function} $\pi(x)$, in particolare la sua proprietà
	\begin{equation*}
	\pi(x)=\pi(\sqrt{x})+\phi(x,\sqrt{x})-1
	\end{equation*}
	Come è immaginabile questa funzione \enquote{conta} il numero di primi minori o uguali di $x$. Si noti che la proprietà notevole di $\pi(x)$ si basa sul fatto che i fattori primi di $x$ sono minori (al più uguali) della sua radice. \\ Allora siano $p_1,\dots, p_s$ i numeri primi minori o uguali ad $y$, per principio di inclusione-esclusione\footnote{la cardinalità dell'unione di due insiemi $A$ e $B$ è la cardinalità di $A$ più quella di $B$ meno la loro intersezione.} otteniamo
	\begin{equation*}
	\phi(x,y)=x-\sum_{1\leq i \leq s}\left[\frac{x}{p_i}\right]+\sum_{1\leq i,j \leq s}\left[\frac{x}{p_ip_j}\right]+ \dots \ + (-1)^s\left[\frac{x}{p_i\dots p_s}\right]
	\end{equation*}
	Se questo fosse l'insieme di tutti e soli i numeri primi potremmo considerare $N=p_1\dots p_s$. Per definizione della funzione di Legendre vale
	\begin{equation*}
	\phi(N^2,N)=1
	\end{equation*}
	ma per quanto abbiamo appena visto dalla formula
	\begin{align*}
	\phi(N^2,N)&=1\\
	&=N^2-\sum_{1\leq i \leq s}\left[\frac{N^2}{p_i}\right]+\sum_{1\leq i,j \leq s}\left[\frac{N^2}{p_ip_j}\right]+ \dots \ + (-1)^s\left[\frac{N^2}{p_i\dots p_s}\right]N^{-1} = mN
	\end{align*}
	per un certo $m$ intero. \\ In sintesi se i primi fossero solo $s$ ogni elemento non nullo sarebbe invertibile; quindi se fosse vero $\mathbb{N}$ sarebbe un campo, cosa che non è vera (ad esempio non è nemmeno un anello commutativo con unità perché l'abeliano $(\mathbb{N}^*,\cdot)$ non ha un inverso moltiplicativo, oppure perché $(\mathbb{N},+)$ non ha un inverso additivo).
\end{proof}




\subsection{Le funzioni aritmetiche su $\mathbb{N}$}
Da adesso in poi useremo per comodità la notazione $(n,m)$ per indicare il massimo comune divisore tra $m$ ed $n$.\\ Nel caso dovesse essere ambigua (ad esempio in presenza del generatore di un ideale) torneremo alla notazione meno compatta.
\begin{definizione}(Funzione aritmetica in letteratura) 
	\enquote{\textit{Dico \textbf{funzione aritmetica} una funzione $f(n)$ definita su tutti gli $n \in \mathbb{N}$, solitamente costruita come funzione a valori complessi}
	\begin{equation*}
	f:\mathbb{N}\longrightarrow\mathbb{C}
	\end{equation*}
	\textit{In molti dei casi più importanti $f(n))$ è un intero che descrive una proprietà di $n$ nella teoria dei numeri.}} (\cite[capitolo 8, pagina 143]{J12}) \\ \\
	\enquote{\textit{Funzioni $f(n)$ sugli interi positivi definite in modo da esprimere una proprietà aritmetica di $n$ sono dette \textbf{funzioni aritmetiche}.}}(\cite[capitolo 16, pagina 133; quarta edizione]{H08})%todo: mette un + a caso nel tag biblio [Har+08]
\end{definizione}
Non sempre è facile definire le funzioni aritmetiche, difatti come abbiamo visto possono essere date definizioni molto diverse ma nello stesso spirito. \\ Per noi una funzione sarà tale solo se avrà valori in $\mathbb{C}$, ma sottintenderemo la sua capacità del rappresentare una proprietà di $n$ (non ci interessa che le funzioni senza tali proprietà siano aritmetiche o meno). 
\begin{definizione}[Funzione aritmetica]
	Diciamo \textbf{funzione aritmetica} una generica
	\begin{equation*}
	f:\mathbb{N}\longrightarrow\mathbb{C}
	\end{equation*}
\end{definizione}
Le funzioni aritmetiche sono fondamentali in teoria dei numeri, perché come abbiamo letto esprimono tramite valori di $f(n)$ alcune proprietà di $n$. \\ Ad esempio una funzione aritmetica può essere
\begin{itemize}
	\item \textbf{Additiva}, $f(n+m)=f(n)+f(m) \ \forall m,n \in \mathbb{N}$ tali che $(n,m)=1$.
	\item \textbf{Completamente additiva}, $f(n+m)=f(n)+f(m) \ \forall m,n \in \mathbb{N}$.
	\item \textbf{Moltiplicativa}, $f(n\cdot m)=f(n)\cdot f(m) \ \forall m,n \in \mathbb{N}$ tali che $(n,m)=1$.
	\item \textbf{Completamente moltiplicativa}, $f(n\cdot m)=f(n)\cdot f(m) \ \forall m,n \in \mathbb{N}$.
\end{itemize}
\begin{proposizione}
	\label{somma_moltiplicativa}
	Se $f$ è una funzione aritmetica moltiplicativa, allora 
	\begin{equation*}
	g(n)\coloneqq \sum_{d\mid n}f(d)
	\end{equation*}
	è una funzione aritmetica moltiplicativa.
\end{proposizione}
\begin{proof}
	Devo semplicemente dimostrare che $g$ è moltiplicativa seguendo la definizione. Siano allora $m,n \in \mathbb{N}$ tali che $(n-m)=1$.
	\begin{align*}
	g(m \cdot n)&=\sum_{d\mid mn}f(d)=\sum_{a\mid m}\sum_{b\mid n}f(a\cdot b)=\sum_{a\mid m}\sum_{b\mid n}f(a)\cdot f(b) \\
	&=\left[\sum_{a\mid m}f(a)\right]\cdot\left[\sum_{b\mid n}f(b)\right]=g(m)\cdot g(n)
	\end{align*}
	Nel secondo passaggio ho usato il fatto che se $d$ divide $mn$, allora $d$ è il prodotto di certi $a$ e $b$ coprimi tali che uno divide $m$ e l'altro divide $n$.
\end{proof}
Vediamo alcune delle più importanti funzioni algebriche.
\paragraph{Funzione di Eulero} \ \\ Detta anche \textit{Euler's totient function}.
\begin{align*}
\varphi: \mathbb{N}&\longrightarrow \mathbb{N}\\
n &\longmapsto |\mathbb{Z}_n^*|
\end{align*}
La proprietà di teoria dei numeri che la funzione di Eulero descrive è (abbastanza ovviamente) la cardinalità dell'insieme
\begin{equation*}
\left\{m\in\mathbb{N} \ \middle| \ 1\leq m \leq n, \ (m,n)=1\right\}
\end{equation*}
\paragraph{Funzione di M\"obius}
\begin{align*}
\mu: \mathbb{N}&\longrightarrow \left\{-1,0,1\right\}\\
n &\longmapsto 
\begin{cases}
1 \  & \text{se $n=1$}\\
0 \  & \text{se $n\neq1$ non è libero da quadrati}\\
(-1)^k \  & \text{se $n\neq1$ è libero da quadrati, $n=p_1\dots p_k$}\\
\end{cases}
\end{align*}
La proprietà descritta è palese; vediamo qualche esempio.
\begin{esempio}
	\begin{equation*}
	\arraycolsep=1.4pt
	\begin{array}{ll}
		\mu(7)&=-1	\\ 
		\mu(18)&=\mu(3^2\cdot2)=0\\
		\mu(30)& =\mu(2\cdot3\cdot5)=-1\\
		\mu(6) &=\mu(2\cdot3)=1
	\end{array}
	\end{equation*}
\end{esempio}
\begin{osservazione}[Proprietà] \
	\begin{enumerate}
		\item Vediamo che la funzione è moltiplicativa. Siano i $p_i$ primi distinti e siano gli $e_i$ esponenti (interi) maggiori di $1$.
		\begin{equation*}
		\mu(p_1\dots p_k)=(-1)^k \iff \mu(p_1)\cdot \ \dots \ \cdot \mu(p_k)=(-1)^k
		\end{equation*}
		\begin{equation*}
		\mu(p_1^{e_1}\dots p_k^{e_k})=0 \iff \mu(p_1^{e_1})\cdot \ \dots \ \cdot \mu(p_k^{e_k})=0
		\end{equation*}
		Nel secondo caso (valori non liberi da quadrati) le due implicazioni sono banali, nel primo basta osservare che necessariamente il prodotto di primi è libero da quadrati se tutti i primi sono diversi tra loro (coprimi e liberi da quadrati).
		\item La funzione non è completamente moltiplicativa.
		\begin{align*}
		\mu(12)&=\mu(2^2\cdot3)=0\\ &\neq \mu(6)\cdot \mu(2)=(1)(-1)
		\end{align*}
		\item La funzione non è additiva.
		\begin{align*}
		\mu(2)&=1\\
		&\neq \mu(1)+\mu(1)=0+0
		\end{align*}
		Siccome il massimo comune divisore tra $2$ ed $1$ è $1$, $\mu$ non è additiva.
		\item La funzione non è completamente additiva. 
		\begin{center}
			Non essendo additiva, ovviamente \\ non può essere completamente additiva.
		\end{center}
		
	\end{enumerate}
\end{osservazione}
\begin{proposizione}
	\label{somma_moebius}
	\begin{equation*}
	\sum_{d\mid n}\mu(d)=
	\begin{cases}
	1 \ & \text{se $n=1$}\\
	0 \ & \text{se $n>1$}
	\end{cases}
	\end{equation*}
\end{proposizione}
\begin{proof} 
	Sia $N=p_1^{e_1}\dots p_k^{e_k}$ sotto le opportune ipotesi (viste prima); siccome per $N=1$ la dimostrazione è banale, sia anche $N>1$.
	\begin{align*}
	\sum_{d\mid n}\mu(d)&=1+\sum_i\mu(p_i)+\sum_{i,j}\mu(p_ip_j)+\dots\\
	&= 1-k+\binom{k}{2}-\binom{k}{3}+\ \dots \ +(-1)^k=(1-1)^k=0
	\end{align*}
	Ho usato pesantemente in entrambi i passaggi la definizione di $\mu$; nel primo è bastato il fatto che non appena un primo è almeno elevato alla seconda la funzione restituisca zero, nel secondo invece si è solo valutata $\mu$.
\end{proof}
\begin{teorema}[Formula di inversione di M\"obius]
	Siano $f$ e $g$ due funzioni aritmetiche. Se vale 
	\begin{equation*}
	f(n)=\sum_{d\mid n}g(d)
	\end{equation*}
	allora possiamo scrivere
	\begin{equation*}
	g(n)=\sum_{d\mid n}\mu\left(\frac{n}{d}\right)f(d)=\sum_{d\mid n}\mu(d)f\left(\frac{n}{d}\right)
	\end{equation*}
\end{teorema}
\begin{proof}
	Iniziamo osservando che dato un divisore $d$ di $n$ (ovvero un valore tale che $n/d$ è intero) possiamo scrivere
	\begin{align*}
	\sum_{d\mid n}\mu\left(\frac{n}{d}\right)f(d)&=\sum_{e\cdot d=n}\mu(e)f(d)=\sum_{e\mid n}\mu(e)f\left(\frac{n}{e}\right)\\
	&=\sum_{d\mid n}\mu(d)f\left(\frac{n}{d}\right)
	\end{align*}
	dato che $e$ è divisore di $n$.
	\begin{align*}
	\sum_{d\mid n}\mu\left(\frac{n}{d}\right)f(d)&=\sum_{d\mid n}\mu\left(\frac{n}{d}\right)\sum_{d'\mid d}g(d')=\sum_{d'\mid d\mid n}g(d')\mu\left(\frac{n}{d}\right)\\
	&=\sum_{d'\mid \frac{n}{m}\mid n}g(d')\mu\left(m\right)
	\end{align*}
	%todo: notazione strana
	dove abbiamo usato le seguenti osservazioni:
	\begin{enumerate}
		\item ho definito $m=\frac{n}{d}$
		\item se $d$ e $d'$ sono divisori di $n$, allora $d'\mid d$ vale se e solo se $\frac{n}{d}\mid \frac{n}{d'}$
		\item se $d'$ ed $\frac{n}{m}$ sono divisori di $n$, allora $d'\mid \frac{n}{m}$ vale se e solo se $m\mid \frac{n}{d'}$
	\end{enumerate}
	Per la proposizione \ref{somma_moebius} sappiamo che nella sommatoria sopravvivono elementi solo per $d'=n$, quindi otteniamo la tesi.
\end{proof}
\paragraph{Funzioni dei divisori}
\begin{equation*}
\sigma_k(n)=\sum_{d\mid n}d^k
\end{equation*}
\begin{equation*}
d(n)=\sum_{d\mid n}1
\end{equation*}
Notiamo che $\sigma_0(n)=d(n)$ e $\sigma_1(n)=\sigma(n)$, la funzione somma dei divisori di $n$. \\ Abbastanza evidentemente $d(n)$ restituisce il numero di divisori di $n$. Invece $\sigma_k(n)$ è la somma delle $k$-esime potenze dei divisori di $n$.
\begin{teorema}
	Le due funzioni sono moltiplicative.
\end{teorema}
\begin{proof}
	Osserviamo che le due funzioni $u(n)=1$ e $\operatorname{id}(n)=n$ sono aritmetiche e moltiplicative. Ma allora
	\begin{equation*}
	d(n)=\sum_{d\mid n}1=\sum_{d\mid n}u(d)
	\end{equation*}
	\begin{equation*}
	\sigma(n)=\sum_{d\mid n}d=\sum_{d\mid n}\operatorname{id}(d)
	\end{equation*}
	e quindi per la proposizione \ref{somma_moltiplicativa} sono moltiplicative.
\end{proof}




\section{Divisibilità nei domini di integrità}
\subsection{Struttura dei domini di integrità}
\label{lezione3}
Iniziamo a generalizzare quanto visto nella sezione precedente, da adesso non ci restringeremo più a $\mathbb{Z}$; lavoreremo su una classe di strutture algebriche più generica.
\begin{definizione}[Dominio di integrità] 
	Sia $A$ un anello commutativo con identità 1, lo dico \textbf{dominio di integrità} se 
	\begin{equation*}
	\forall a,b \in A. \ a \cdot b = 0 \implies a=0 \text{ oppure } b=0
	\end{equation*}
\end{definizione}
\begin{definizione}
	\
	\begin{enumerate}
		\item Un $e \in A$ si dice \textbf{unità} se è invertibile rispetto al prodotto. \\ Solitamente viene indicata con u, in letteratura si trova anche $e$.
		\item Degli $u,v \in A$ si dicono \textbf{associati} se $u=ev$.
		\item Un $q \in A$ si dice \textbf{irriducibile} se è non-nullo, se non è un'unità e se 
		\begin{equation*}
		\forall a,b \ \text{con} \ q=ab, \ \text{$a$ oppure $b$ sono un'unità}
		\end{equation*}
		\item Un $p \in A$ si dice \textbf{primo} se $p\mid ab \implies $ $p\mid a$ oppure $p\mid b$. \footnote{nella definizione compare un \textit{vel}, un \textit{aut inclusivo}; possono essere vere entrambe. Si pensi a $p\mid p\cdot p$.}
	\end{enumerate}
\end{definizione}
\begin{osservazione} 
	Facciamo alcune brevi osservazioni sulle definizioni appena date. Ricordiamo che per $k \in A$ anello commutativo possiamo scrivere in notazione compatta 
	\begin{equation*}
	(k) \coloneqq \left\{ak \ \middle| \ a \in A\right\}
	\end{equation*}
	\begin{enumerate}
		\item \textbf{$a$ divide $ b \ \iff \ (b) \subset (a)$}\\
		Se $a\mid b$ questo vuol dire che $b=ac$ per un certo $c \in A$. Allora sia dato un $x \in (b)$, ha forma $x=kb$, ma per ipotesi $x=k(ac)=(kc)a$ quindi $x$ appartiene anche all'ideale generato da $a$. Si noti come abbiamo usato le ipotesi di $A$ anello commutativo.\\
		Implicazione inversa: per ipotesi $b \in (a)$, quindi $b=kc$ per un certo $k \in A$.
		\item \textbf{$a,b$ associati $\iff \ (a)=(b)$}\\
		Sia $u$ unità. Se sono associati allora $a=1b$, prendo $x \in (a)$; $x=ka$ per un certo $k$, ma per ipotesi $x=k(ub)=b(uk)$ ed $x \in (b)$. Manca da dimostrare $(b) \subset (a)$, ma posso procedere allo stesso modo ricordando che $u$ è invertibile, quindi 
		\begin{equation*}
		a=ub \implies au^{-1}=b
		\end{equation*}
		e procedo ugualmente. \\ 
		Se invece $(a)=(b)$, questo vuol dire che $a=k_ab$ e $b=k_ba$. 
		\begin{equation*}
		a=a(k_a\cdot k_b) \implies k_a\cdot k_b=1 
		\end{equation*}
 		ma allora posso dire (senza perdita di generalità) che $x$ è l'unità, $a$ e $b$ sono associati.
		\item \textbf{$p$ è primo $\iff$ $(p)$ è primo}\\
		Ricordiamo che un ideale $I$ si dice primo se è proprio (non è $A$ stesso) e $\forall a,b$ se $a,b\notin I$ allora $ab \notin I$. Ovviamente possiamo invertire l'implicazione negando i due lati. 
		\begin{equation*}
		ab \in (p) \implies a \in (p) \ \text{oppure} \ b \in (p)
		\end{equation*}
		Se l'ideale è primo prendiamo $ab \in (p)$, questo vuol dire che $ab=kp$; inoltre per ipotesi se $ab\in(p)$ questo vuol dire che $a\in(p)$ oppure $b\in(p)$.
		Ma vediamo che queste due ipotesi sono $a=k_ap$ e $b=k_bp$, che sono esattamente quanto volevo dimostrare. \\ L'implicazione opposta è banalmente la stessa dimostrazione ma ripercorsa al contrario.
	\end{enumerate}
	Si noti infine che quanto definiamo \textit{unità} non è l'unità intesa quando si definisce un \textit{anello commutativo con unità 1}; i primi sono gli invertibili rispetto al prodotto, la seconda è la classica \textit{identità moltiplicativa} - che indicheremo in notazione con 1. Chiaramente l'identità moltiplicativa è per sua definizione una unità, ma per l'identità moltiplicativa abbiamo delle buone proprietà, ad esempio l'unicità (è facile dimostrarlo), che come si è visto non vale per le unità.
\end{osservazione}
\begin{esempio}\
	\begin{enumerate}
		\item In $\mathbb{Z}$ le unità sono $\pm 1$, quindi per ogni $a \in \mathbb{Z}$ gli associati sono $\pm a$.
		\item Se $\mathbb{K}$ è un campo e prendo $\mathbb{K}[x]$, i polinomi costanti $a$ ($a \in \mathbb{K}^*$) sono unità e dato $f(x) \in \mathbb{K}[x]$ gli associati hanno forma $a\cdot f(x)$ per $a \in \mathbb{K}^*$.
	\end{enumerate}
\end{esempio}
\begin{osservazione} 
	\label{notinD}
	In un dominio $A$ qualunque un elemento primo è irriducibile. Basta replicare la dimostrazione che abbiamo visto in precedenza.\\
	Se $p$ è primo e $p=ab$, allora $p\mid a$ oppure $p\mid b$. Supponiamo per esempio che valga la prima, allora 
	\begin{equation*}
	p=ab=(pk)b \implies kb=1 
	\end{equation*}
	e quindi $b$ è l'unità. \\ 
	Per l'implicazione inversa non basta essere in un dominio, infatti avevamo usato Bezout e la divisione euclidea.
\end{osservazione}
\begin{definizione}
	Sia $A$ un anello commutativo con unità 1.
	\item Un suo ideale $I$ si dice \textbf{massimale} se è proprio ($I \neq A$) ed ogni ideale $J$ di $A$ con $I \subset J$ è improprio.
	\item Un suo ideale $I$ si dice \textbf{primo} se è proprio ($I \neq A$) e $\forall a,b \in A$ se $a,b$ non sono in $I$ non vi appartiene neanche il loro prodotto.
	\item Un suo ideale $I$ si dice \textbf{principale} se è generato da un unico elemento, ovvero $I=(a)$ per un certo $a \in I$.
\end{definizione}
\begin{definizione}
	Sia $A$ un dominio.
	\begin{enumerate}
		\item $A$ si dice \textbf{dominio ad ideali principali (PID)} se ogni suo ideale è principale.
		\item $A$ si dice \textbf{dominio a fattorizzazione unica (UFD)} se ogni suo elemento non invertibile e non nullo ammette una fattorizzazione in elementi riducibili, unica a meno di riordinamento e di elementi associati (\textit{fattorizzazione essenzialmente unica}).
		\item $A$ si dice \textbf{euclideo} se esiste una funzione 
		\begin{equation*}
		f: A^*\longrightarrow \mathbb{N}
		\end{equation*}
		tale che per ogni $a,b\in A$ con $b \neq 0$ esistono $q,r \in A$ con $a=qb+r$ e si ha $r=0$ oppure $f(r)\leq f(b)$. \\
		Intuitivamente si dice tale perché è possibile effettuare la \textit{divisione euclidea} al suo interno.
		\item $A$ si dice \textbf{principalmente noetheriano} se ogni successione crescente di ideali principali è stazionaria; ovvero se data la catena di inclusioni di ideali principali
		\begin{equation*}
		(a_1)\subset(a_2)\subset \dots
		\end{equation*}
		esiste un $n\in\mathbb{N}$ tale che $(a_n)=(a_{n+1})=\dots$
	\end{enumerate}
\end{definizione}
\begin{esempio}
	Un esempio di dominio euclideo è $\mathbb{Z}$, a livello intuitivo anche solo banalmente per la possibilità di effettuare la divisione euclidea al suo interno. \\ Se riflettiamo su come questa avviene possiamo anche dimostrare formalmente che sia euclideo: per come abbiamo definito nella scorsa sezione l'algoritmo euclideo basta porre $f(k)=|k|$. \\ \\ Altro esempio meno banale è $\mathbb{K}[x]$ con $\mathbb{K}$ campo, per effettuare la divisione tra i suoi elementi (\textit{polinomi in $x$}) è sufficiente porre $f(p)=\deg(p)$. \\ Si noti che può essere molto pericoloso definire il grado del polinomio costantemente nullo, ma non è un problema perché la scrittura formale per la divisione euclidea tra polinomi è 
	\begin{equation*}
	p=nq+r \ \ \ \text{ con } \deg(r)<\deg(q) \ \text{ oppure } r=0
	\end{equation*}
\end{esempio}
\subsection{Fattorizzazione unica e noetherianità}
\begin{osservazione}[Come siamo arrivati qui?]
	Noi vorremo lavorare nei domini a fattorizzazione unica. Abbiamo visto nell'Osservazione \ref{notinD} che la struttura di dominio di integrità \textit{non basta} per lavorare in uno spazio \enquote{buono}, cioè uno che goda di fattorizzazione unica come $\mathbb{Z}$. \\ L'ingrediente fondamentale saranno le catene di ideali principali nella definizione di anello principalmente noetheriano, come vedremo subito.
\end{osservazione}
\begin{teorema}[Esistenza della fattorizzazione]
	Se $A$ è principalmente noetheriano allora posso fattorizzare ogni suo elemento non-nullo e diverso da un'unità come prodotto di elementi irriducibili con \textit{fattorizzazione essenzialmente unica}.
\end{teorema}
\begin{proof}
	Sia $a$ riducibile, quindi $a=xy$. Costruiamo una catena con il seguente algoritmo, poi potremo usare l'ipotesi di noetherianità.
	\begin{itemize}
		\item se $x$ non è irriducibile allora $x=a_1d_1$ con $a_1 \neq 0$ ed $a_1,d_1$ non unità:
		\begin{equation*}
		a_1\mid x \mid a \implies (a)\subset(x)\subset(a_1)
		\end{equation*}
		(se invece è irriducibile mi fermo, avrei già una catena stazionaria)
		\item se $a_1$ non è irriducibile allora $x=a_2d_2$ con $a_2 \neq 0$ ed $a_2,d_2$ non unità:
		\begin{equation*}
		(a)\subset(x)\subset(a_1)\subset(a_2)
		\end{equation*}
		\item e così via, ottenendo
		\begin{equation*}
		(a)\subset(x)\subset(a_1)\subset(a_2)\subset(a_3)\subset\dots
		\end{equation*}
	\end{itemize}
	Per ipotesi esiste un $n$ tale che $(a_n)=(a_{n+1})$, ovvero $a_{n}$ ed $a_{n+1}$ sono associati:
	\begin{equation*}
	a_{n}=ua_{n+1} \ \ \ \ \text{ con $u$ unità}
	\end{equation*}
	ed inoltre da costruzione abbiamo $a_{n}=d_{n+1}a_{n+1}$ non unità e quindi
	\begin{equation*}
	a_{n+1}(d_{n+1}-u)=0 \implies d_{n+1}=u
	\end{equation*}
	che implica $a_n$ irriducibile. Ripetendo il ragionamento per $y$ ottengo la tesi. \\ Abbiamo infatti dimostrato che $a$ si scrive tramite prodotto tra un fattore irriducibile (quello che abbiamo trovato) ed \enquote{altro}, ma possiamo fattorizzare la parte non irriducibile, per ipotesi di noetherianità ci dovremo fermare ottenendo un prodotto in irriducibili, essenzialmente unico. \\ \\ Notiamo che la definizione di principalmente noetheriano svolge la funzione del \textit{principio del minimo} sui numeri naturali, equivalente al principio di induzione.\footnote{questo vale per la costruzione dei naturali tramite gli \textit{assiomi di Peano}.} Ci siamo ricondotti elegantemente ad una situazione analoga di quella studiata nella fattorizzazione unica degli interi.
\end{proof}
	\label{lezione4}
\begin{teorema}[Unicità della fattorizzazione]
	Le seguenti sono equivalenti.
	\begin{enumerate}
		\item $A$ è principalmente noetheriano ed ogni elemento irriducibile è primo
		\item ogni $a \in A$ con $a\neq0$ e non unità si scrive come prodotto di numeri primi (ed eventualmente una unità)
		\item $A$ è un UFD
	\end{enumerate}
\end{teorema}
\begin{proof}\
	\begin{itemize}
		\item[$1 \implies 2$] Vale per il teorema precedente.
		\item[$2 \implies 3$] Devo mostrare che la fattorizzazione è unica. Siamo $p_i$ e $q_j$ primi (e quindi irriducibili) con
		\begin{equation*}
		a=p_1 \dots p_r = q_1 \dots q_s \ \ (r<s)
		\end{equation*}
		allora $p_1\mid q_1 \dots q_s$, che a sua volta implica $p_1\mid q_1$. Essendo irriducibili sono associati, quindi $p_1=u_1 q_1$ con $u_1$ unità. Procedendo in questo modo per $r$ volte ottengo
		\begin{equation*}
		1=\xi_1\dots\xi_r q_{r+1}\dots r_s
		\end{equation*}
		e quindi $r=s$. Infatti se non lo fosse tutti i primi a destra sarebbero invertibili. La fattorizzazione è unica a meno di associati.
		\item[$3 \implies 1$] Dimostriamo prima di tutto che ogni irriducibile è primo. \\ Sia $q\mid ab$, ovvero $ab=qd=\alpha$.
		\begin{equation*}
		d=m_1\dots m_t, \ \ a=p_1\dots p_r, \ \ b=q_1\dots \ q_s
		\end{equation*}
		\begin{equation*}
		\alpha = q (m_1\dots m_t)=(p_1\dots p_r)(q_1\dots \ q_s)
		\end{equation*}
		Per unicità della fattorizzazione $q\mid p_i$ oppure $q\mid q_j$, cioè $q\mid a$ oppure $q\mid b$. \\ \\
		Dimostriamo che $A$ è principalmente noetheriano. \\ Prendo una catena $(a_1)\subset(a_2)\subset\dots$, utilizzo le fattorizzazioni.
		\begin{itemize}
			\item $a_1=q_1\dots q_r$
			\item siccome $a_2 \mid a_1$ ogni primo della fattorizzazione di $a_2$ compare in quella di $a_1$, inoltre $a_1 \neq a_2$, quindi $a_2$ ha meno fattori
			\item procedo iterativamente, arrivo ad un $a_n$ irriducibile con 
			\begin{equation*}
			(a_n)=(a_{n+1})=\dots
			\end{equation*}
		\end{itemize}
	\end{itemize}
\end{proof}

Questa condizione però non è \enquote{soddisfacente}, infatti la principale noetherianità non è sempre semplice da verificare. Proviamo a costruire ulteriori relazioni tra i tipi di domini che abbiamo identificato. \\ Ad esempio potremmo caratterizzare i domini in cui gli irriducibili sono anche primi.
\begin{teorema}[Caratterizzazione dei PID]
	Se $A$ è un dominio ad ideali principali (PID) le seguenti sono equivalenti.
	\begin{enumerate}
		\item $p$ è primo in $A$
		\item $p$ è irriducibile in $A$
		\item $p\in A$, $(p)$ è massimale
	\end{enumerate}
\end{teorema}
\begin{proof}\
	\begin{itemize}
		\item[$1 \implies 2$] Sia $p=ab$, voglio dimostrare che uno dei due è l'unità. Poiché $p$ è primo, $p\mid a$ oppure $p\mid b$. Scegliamo la prima senza perdita di generalità, allora 
		\begin{equation*}
		a=pc \implies p=(pc)b\implies cb=1 \implies b \text{ unità}
		\end{equation*}
		\item[$2 \implies 3$] sia $(p)\subset I$ con $I$ ideale di $A$, poiché $A$ è un PID allora $I=(k)$ per un certo $k$. Ma come abbiamo visto questo implica che $k\mid p$.
		\begin{equation*}
		k\mid p \implies p=kc \implies k \text{ oppure } c \text{ unità}
		\end{equation*}
		Se $k$ è unità allora $(k)=A$ perché conterrebbe 1.\footnote{se un ideale di $A$ contiene 1 è per sua stessa definizione un ideale improprio, cioè $A$; la verifica è banale.} \\ Se invece $c$ fosse unità potrei scrivere $k=c^{-1}p$ per ottenere $(k)=(p)$.
		\item[$3 \implies 1$] Se $(p)$ è massimale allora è primo, ma abbiamo visto che questo implica $p$ primo.
	\end{itemize}
\end{proof}
\begin{teorema}[Caratterizzazione degli UFD]
	Se $A$ è un PID, allora è un dominio a fattorizzazione unica (UFD).
\end{teorema}
\begin{proof}
	Dimostriamo che in un PID ogni ideale è finitamente generato. \\ Consideriamo una catena di ideali $I_1\subset I_2 \subset \dots$, allora $I=\bigcup_{n=1}^{+\infty}I_n$ è un ideale. Normalmente sarebbe problematico approcciarsi ad un oggetto simile (ha senso scrivere una unione infinita? è ancora un ideale? come scrivo i suoi generatori?), ma sappiamo che $A$ è un dominio ad ideali principali! Allora  
	\begin{equation*}
	I = (a_1, \dots, a_m)
	%todo: come fa ad essere principale se ha m elementi?
	\end{equation*}
	 e quindi esiste $N>0$ tale che $I_N$ contenga i generatori di $I$, questo vale necessariamente per come ho costruito $I$, per come ho scelto gli ideali e per proprietà di $A$. Ho appena ottenuto che $I \subset I_N$, quindi evidentemente $I_n=I_N$ per ogni $n\geq N$. \\ Allora $A$ è principalmente noetheriano, e siccome è ad ideali principali ogni elemento irriducibile è primo. Allora è un dominio a fattorizzazione unica.
\end{proof}
\begin{teorema}[Caratterizzazione degli euclidei] 
	Se $A$ è un dominio euclideo, allora è un dominio ad ideali principali (PID).
\end{teorema}
\begin{proof}
	Sia $I$ un ideale di $A$.
	\begin{equation*}
	X \coloneqq \left\{f(a) \ \middle| \ a\in I, a \neq 0\right\}\subset \mathbb{N}
	\end{equation*}
	Chiaramente $X$ è non vuoto, ammette un minimo ed un massimo in $\mathbb{N}$. \\ Sia $n_0$ il minimo, sia $a_0 \in I$ tale che $f(a_0)=n_0$. Ora prendo un $b \in I$ con $b=a_0q+r$ e $r=b-a_0q\in I$; quindi $r=0$. Allora $b=aq$ ed $I=(a)$.
	%todo: rivedi dimostrazione ed enunciato, possibile errore
\end{proof}
\begin{esempio}
	Possiamo usare i potenti strumenti che abbiamo appena costruito per studiare il caso \enquote{banale} di $\mathbb{Z}$ con facilità, ed addirittura espanderci a casi ben più complessi.
	\begin{enumerate}
		\item $\mathbb{Z}$ è euclideo, quindi un PID, quindi un UFD.
		\item $\mathbb{K}[x]$ è euclideo, quindi un PID, quindi un UFD.
	\end{enumerate}
\end{esempio}
\begin{osservazione}[Cosa ho appena letto?]
	\
	\begin{enumerate}
		\item in un dominio di integrità un primo è irriducibile, ma non sempre si ha il viceversa
		\item i domini principalmente noetheriani hanno la fattorizzazione in elementi irriducibili; se gli irriducibili sono primi allora la fattorizzazione è anche unica
		\item in un PID un elemento è primo se e solo se è irriducibile, inoltre un PID è principalmente noetheriano 
		\begin{itemize}
			\item in generale se gli ideali sono finitamente generati ogni catena ascendente è stazionaria
			\item un PID è un UFD
		\end{itemize}
	\item un dominio Euclideo è un PID, che a sua volta è un UFD
	\end{enumerate}
\end{osservazione}




\subsection{L'anello degli interi di Gauss}
\label{lezione5}
Possiamo sfruttare la teoria appena costruita per studiare una classe molto interessante di anelli. \\
Consideriamo l'insieme
\begin{equation*}
\mathbb{Z}[\sqrt{-k}]\coloneqq \left\{a+b\sqrt{-k} \, \mid \, a,b\in\mathbb{Z}\right\}
\end{equation*}
per $k$ un naturale positivo. Spesso lo indicheremo come $\mathbb{Z}[\omega]$, $\omega = \sqrt{-k}=i\sqrt{k}$. \\ Si tratta di un sottoanello di $\mathbb{C}$, quindi di un dominio di integrità.
\begin{definizione}[Interi di Gauss] 
	Per $\omega=\sqrt{-1}=i$ otteniamo \textbf{l'anello degli interi di Gauss} $\mathbb{Z}[i]$. I suoi elementi hanno forma $z=a+ib$ con $a,b\in\mathbb{Z}$.\footnote{si possono immaginare gli interi di Gauss come il \enquote{reticolo intero} del piano complesso.}
\end{definizione}
\begin{osservazione}[Le unità negli interi di Gauss] 
	Quali sono le unità in $\mathbb{Z}[i]$? Studiamone la struttura, stiamo cercando gli invertibili rispetto al prodotto.
	Dato $z=a+ib\in\mathbb{C}$, si verifica facilmente che il suo inverso moltiplicativo è $w=c+di\in\mathbb{C}$ con
		\begin{equation*}
			c=\frac{a}{a^2+b^2} \ \ \ \ \ d = \frac{-b}{a^2+b^2}
		\end{equation*}
		quindi perché l'inverso di $z\in\mathbb{Z}[i]$ appartenga di nuovo a $\mathbb{Z}[i]$ è necessario che $c$ e $d$ siano interi (ovvero che $a$ e $b$ siano multipli di $a^2+b^2$):
		\begin{equation*}
			a=c(a^2+b^2) \ \ \ \ \ b=(-d)(a^2+b^2)
		\end{equation*}
		Ma svolgendo i calcoli
		\begin{align*}
			a^2+b^2
			&=c^2(a^2+b^2)^2+(-d)^2(a^2+b^2)^2 \\ 
			&=(c^2+d^2)(a^2+b^2)^2
		\end{align*}
		Siccome siamo in un dominio di integrità ($\mathbb{C}$) e tutti gli elementi sono di $\mathbb{Z}$ posso scrivere
		\begin{equation*}
			(b^2+d^2)(a^2+b^2)=1 \implies a^2+b^2=1
		\end{equation*}
		e quindi ho quattro casi per $a$ e $b$, poi determino $c$ e $d$ dalla formula.
		\begin{itemize}
			\item $a=1$, $b=0$, $c=1$, $d=0$
			\item $a=-1$, $b=0$, $c=-1$, $d=0$
			\item $a=0$, $b=1$, $c=0$, $d=1$
			\item $a=0$, $b=-1$, $c=1$, $d=-1$
		\end{itemize}
	Abbiamo quindi visto che:
		\begin{enumerate}
		\item Le unità sono $\pm i$ e $\pm 1$.
		\item Essendo sottoanello di un dominio di integrità (ovvio), $\mathbb{Z}[i]$ è banalmente un dominio di integrità.
	\end{enumerate}
	Posso ottenere in modo più algebrico lo stesso risultato?
\end{osservazione}
\begin{definizione}
	Sia $z=a+b\omega\in\mathbb{Z}[\omega]$, $\omega=\sqrt{-k}$ per un $k$ intero positivo; definisco la \textbf{norma di $z$} come 
	\begin{equation*}
	N(z)=z\cdot\overline{z}=a^2+kb^2
	\end{equation*}
\end{definizione}
\begin{osservazione}
	Notiamo che la norma così definita non è una funzione aritmetica; è comunque moltiplicativa.
\end{osservazione}
\begin{proposizione}[Caratterizzazione della norma]\
	\begin{enumerate}
		\item $N(z)=0$ se e solo se $z=0$
		\item $N(z\cdot w)=N(z)\cdot N(w)$ per ogni $z,w\in\mathbb{Z}[\omega]$
		\item $z\in\mathbb{Z}[\omega]$ è una unità se e solo se $N(z)=1$
	\end{enumerate}
\end{proposizione}
\begin{proof}\
	\begin{itemize}
		\item Se $z=0$ ha chiaramente norma nulla, vediamo l'implicazione inversa. Se $a^2+kb^2=0$ per $k$ positivo, $a$ e $b$ devono essere necessariamente nulli.
		\item Si dimostra con calcoli brutali sviluppando $zw$ e $z,w$. Oppure sperando di convincersi abbastanza per non fare i calcoli.
		\item Se $z$ ha norma 1, $z\cdot\overline{z}=1$ e quindi è ovviamente invertibile con inverso il suo coniugato.\\
		Se invece è una delle unità, questo vuol dire che esiste un $x$ con $xz=1$. 
		\begin{equation*}
			N(z)N(x) = N(zx)=N(1)=1
		\end{equation*}
		ed allora $N(z)=1$.
	\end{itemize}
\end{proof}
\begin{osservazione}
	Sia $z=a+b\omega\in\mathbb{Z}[\omega]$ di norma unitaria, questo vuol dire che:
	\begin{itemize}
		\item[$(k>1)$] $b=0$ ed $a=\pm1$
		\item[$(k=1)$] caso degli interi di Gauss
	\end{itemize}
\end{osservazione}
\begin{teorema}
	Ogni $z\in\mathbb{Z}[\omega]$ non nullo si fattorizza in elementi irriducibili.
\end{teorema}
\begin{proof}
	Siano
	\begin{equation*}
		S=\left\{z\neq0, z \in \mathbb{Z}[\omega] \ \text{che non si fattorizzano in irriducibili}\right\}
	\end{equation*}
	\begin{equation*}
		X=\left\{N(z) \, \mid \, z \in S\right\}\subset\mathbb{N}\setminus\left\{0\right\}
	\end{equation*}
	Suppongo $X$ non vuoto, allora ammette un elemento minimo; sia $w$ l'elemento di $S$ per cui $X$ ha tale minimo, in simboli $N(w)=\min X$. \\ Essendo non irriducibile posso scrivere $z=xy$ con né $x$ né $y$ unità.
	\begin{equation*}
	N(z)=N(x)N(y) \ \text{con} \ N(x),N(y)>1 
	\end{equation*}
	Per scelta di $z$ si vede che $x$ ed $y$ non appartengono ad $S$, ma allora si fattorizzano in prodotto di elementi irriducibili, e di conseguenza anche $z$. \\ Allora l'insieme $S$ è vuoto.
\end{proof}
\begin{proposizione}
	Se $k\geq3$ la fattorizzazione in irriducibili esiste (per teorema visto), ma non è unica.
\end{proposizione}
\begin{proof}
	Basta trovare un elemento irriducibile che non sia primo; infatti in un UFD un elemento è irriducibile se e solo se è un primo. \\
	Iniziamo osservando che 2 è irriducibile. Sia $xy=2$, allora $N(x)N(y)=4$: 
	\begin{itemize}
		\item se $N(x)=1$, allora $x$ è una unità
		\item se $N(x)=2$ allora $N(y)=2$, ma 
		\begin{equation*}
		2=N(x)=N(a+b\omega)=a^2+kb^2
		\end{equation*}
		non ha soluzioni intere per $k\geq 3$.
	\end{itemize}
	Dividiamo il resto della dimostrazione in due casi:
	\begin{itemize}
		\item[($k$ pari)] $k=2c=(i\sqrt{k})(-i\sqrt{k})$\\ 2 divide $k=2c$ ma nessuno dei due fattori; si può vedere scrivendo la definizione di divisore. 2 è irriducibile e non primo.
		\item[($k$ dispari)] $k+1=2c=(1+i\sqrt{k})(1-i\sqrt{k})$\\ Analogamente 2 divide $k+1$ ma non i suoi due fattori
	\end{itemize}
\end{proof}
\begin{proposizione}
	Se $k=1,2$ allora $\mathbb{Z}[\omega]$ è euclideo (quindi un UFD).
\end{proposizione}
\begin{proof}
	La norma svolge il ruolo della funzione $f$ nella definizione di dominio euclideo.\\ Siano $z,w\in\mathbb{Z}[\omega]$, voglio mostrare che posso fare la divisione intera.
	Se provassimo a calcolare $z/w$ nel piano complesso otterremmo 
	\begin{equation*}
	\frac{z}{w}=x+y\omega \ \ \ \ x,y\in\mathbb{R}
	\end{equation*}
	ma questo può non appartenere a $\mathbb{Z}[\omega]$. \\ Prendo $v=a+b\omega$ elemento più vicino alla frazione, visto come punto nel piano di Gauss. Allora posso stimare che 
	\begin{equation*}
	|x-a|\leq\frac{1}{2} \ \ \ \ \ \ |y-b|\leq\frac{1}{2}
	\end{equation*}
	\begin{equation*}
	\operatorname{d\left(\frac{z}{w},v\right)}=\sqrt{(x-a)^2+(y-b)^2k}\leq\frac{\sqrt{1+k}}{2}<1
	\end{equation*}
	dove ho usato il fatto che $k<3$. \\ Pongo $t=(x-a)+(y-b)\omega$, questo elemento ha norma minore di 1 per quanto appena osservato. Posso prendere $q=v$ ed $r=z-wq$, posso mostrare che ho appena costruito la divisione con resto. Infatti
	\begin{align*}
	N(r)&=N(z-wq)=N\left(w(t+q)-qw\right)\\
	&=N(tw)=N(t)N(w)<N(w)
	\end{align*}
\end{proof}
\subsection{Alcuni esercizi}
%todo: più esercizi
% Alcuni esercizi: 
% Dire per quali $k \in \mathbb{N}$ l'estensione
% $\mathbb{Z}\left[\sqrt{k}\right]$ è un UFD, per quali è  
% anche un PID, per quali invece è anche un anello euclideo.
% Trovare casi di UFD non PID e casi PID non euclidei. 
% Se non si ha voglia di faticare, basta andare sul manuale di
% Algebra A+B di De Graaf (una lista completa dei PID non UFD
% mi pare sia in uno dei pdf di materiale extra dato da
% Murru).
% danke (s)

\begin{esercizio}[Studio di un anello]
Lavoriamo in $\mathbb{Z}[\sqrt{-5}]$. \\ Vogliamo verificare che 3 e $4+\omega$ ($\omega=\sqrt{-5}$) sono irriducibili, primi tra loro e \textit{non vale l'identità di Bezout}. \\ \\
\begin{itemize}
	\item $N(3)=9$; sia $3=xy$, allora $N(x)N(y)=9$. Ponendo $x=a+b\omega$
	\begin{equation*}
	3=N(x)=a^2+5b^2
	\end{equation*}
	che non ha soluzioni intere.
	\item $N(4+\omega)=21$; sia $4+\omega=xy$, allora $N(x)N(y)=21=(3)(7)$. Se lavoriamo come prima otteniamo $N(x)=3$ ed $N(x)=7$ non danno soluzioni intere.
	\item Hanno divisori comuni? Notiamo che le unità nel nostro dominio sono $\pm1$.
	\begin{equation*}
	3=(1)(3)=(-1)(-3)
	\end{equation*}
	\begin{equation*}
	4+\omega=(1)(4+\omega)=(-1)(-4-\omega)
	\end{equation*}
	Evidentemente i due non si dividono a vicenda, infatti se (ad esempio) 3 dividesse $4+\omega$ potrei scrivere
	\begin{equation*}
	(4+\omega)=3(a+b\omega) \ \ \ \text{per certi $a,b\in\mathbb{Z}$}
	\end{equation*}
	che sicuramente non vale. Se invece $4+\omega$ dividesse 3 potrei scrivere
	\begin{equation*}
	(4+\omega)(a+b\omega)=3 \ \ \ \text{per certi $a,b\in\mathbb{Z}$}
	\end{equation*}
	e svolgendo si trova un sistema senza soluzioni intere.
	\item Vale l'identità di Bezout? Ovvero, esistono $x=x_1+x_2\omega$ ed $y=y_1+y_2\omega$ tali che $x_i,y_i\in\mathbb{Z}$ e $3x+(4+\omega)y=1$?\\ Se esistessero tali $x$ ed $y$ avrei
	\begin{equation*}
		\begin{cases}
			3x_1+4y_1-5y_2=1\\
			3x_2+y_1+4y_2=0
		\end{cases}
	\end{equation*}
	da cui $3(x_1-4x_2-7y_2)=1$, che non ha soluzioni intere.
\end{itemize}
%todo: quando non vale Bezout? A volte posso fare la divisione euclidea e non avere Bezout?
\end{esercizio}
\begin{esercizio}[Non unicità di fattorizzazione]
	Studiamo la fattorizzazione in $\mathbb{Z}[\sqrt{-5}]$ di $z=6$; sappiamo che qui la fattorizzazione non è unica, possiamo ottenerne due di $z$?
	\begin{equation*}
	6 =(3)(2) =(1+\omega)(1-\omega)
	\end{equation*}
	e posso vedere che sono tutti irriducibili. \\ Possiamo sfruttare il fatto che $(a-\omega)(a+\omega)=a^2+5$ per ogni $a$ intero, posso ottenere molte fattorizzazioni non uniche sfruttando questo trucco.
	\begin{equation*}
		\begin{array}{lllr}
			\text{per $a=2$} & & &
				9=(2+\omega)(2-\omega)=(3)(3)\\
			\text{per $a=3$} & & &
				14=(3+\omega)(3-\omega)=(2)(7)\\
			\text{per $a=4$} & & &
				21=(4+\omega)(4-\omega)=(3)(7)		
		\end{array}
	\end{equation*}
\end{esercizio}
\begin{esercizio}[Irriducibile non primo]
	Possiamo verificare che 2 è irriducibile e non primo in $\mathbb{Z}[\sqrt{-5}]$. 
	\begin{itemize}
		\item[(irriducibile)] $N(2)=4$, come abbiamo visto più volte se $2=xy$ allora i fattori possono avere norma unitaria (ed allora sono una unità) oppure norma due (ma questo nel nostro anello non è possibile)
		\item[(non primo)] 2 divide $6=(1-\omega)(1+\omega)$ ma non i due fattori
	\end{itemize}
Osserviamo che l'ideale generato da 2 non è primo in $\mathbb{Z}[\sqrt{-5}]$ (in $\mathbb{Z}$ sì), in particolare $\mathbb{Z}[\sqrt{-5}]/(2)$ non è un dominio di integrità.
%todo: aggiungere spiegazione del fatto, forse anche come osservazione o esercizio
\end{esercizio}




\section{Aritmetica modulare}
\label{lezione6}
\subsection{La struttura di $\mathbb{Z}_n$}
L'aritmetica modulare lavora negli anelli delle classi di resto
\begin{equation*}
\left(\mathbb{Z}_n,+,\cdot\right)
\end{equation*}
\begin{osservazione}[Costruzione di un anello di classe di resto]\
	Iniziamo definendo la relazione su $\mathbb{Z}$ 
	\begin{equation*}
	a\sim b \iff n \mid (a-b) 
	\end{equation*}
	per $n$ intero fissato. Ovviamente è una relazione di equivalenza, basta sfruttare le solite proprietà dei divisori. 
	\begin{itemize}
		\item[(riflessiva)] Chiaramente.
		\item[(simmetrica)] Altrettanto ovvio, $n\mid(a-b)$ se e solo se $(a-b)=kn$ per un certo $k$ intero; per invertire la relazione basta prendere $-k$.
		\item[(transitiva)] $n\mid(a-b)$ ed $n\mid(b-c)$ se e solo se $(a-b)=pn$, $(b-c)=qn$ per $p$ e $q$ interi. Dobbiamo ottenere $(a-c)=kn$ per un certo $k$. Dalla prima equazione si vede che $b=a-pn$, sostituendo nella seconda trovo immediatamente
		\begin{equation*}
		(a-pn-c)=qn\implies (a-c)=qn+pn=(q+p)n
		\end{equation*}
	\end{itemize}
	Quindi dire che $a$ e $b$ sono in relazione equivale a dire che hanno lo stesso resto nella divisione euclidea per $n$. \\ Questo caratterizza l'insieme delle classi di equivalenza:
	\begin{equation*}
	\mathbb{Z}_n \simeq \sfrac{\mathbb{Z}}{\sim}
	\end{equation*}
\end{osservazione}
\begin{osservazione}[Costruzione algebrica]
	Possiamo vedere una costruzione più \enquote{algebrica}. Siccome $\mathbb{Z}$ è un PID i suoi ideali sono principali, e sappiamo che hanno forma $(n)=n\mathbb{Z}$. \\ Allora posso costruire la classe di resto tramite le classi laterali
		\begin{equation*}
		\mathbb{Z}_n\simeq\sfrac{\mathbb{Z}}{(n)}\simeq\sfrac{\mathbb{Z}}{n\mathbb{Z}}
		\end{equation*}
		Si vede che $a+(n)=b+(n)$ se e solo se $(a-b)\in(n)$; due elementi appartengono allo stesso ideale se e solo se la loro differenza appartiene al laterale. Quindi le costruzioni sono equivalenti.

\end{osservazione}
\paragraph{Struttura additiva}
\begin{proposizione}
	$\left(\mathbb{Z}_n,+\right)$ è un gruppo abeliano ciclico.
\end{proposizione}
\begin{proof}
	Ovvio che sia un gruppo abeliano, altrettanto ovvio che sia ciclico; basta pensare alla struttura in classi di equivalenza.
\end{proof}
\begin{proposizione}
	Ogni gruppo ciclico di ordine $n$ è isomorfo a $\left(\mathbb{Z}_n,+\right)$.
\end{proposizione}
\begin{proof}
	Se $g$ è ciclico allora è generato da un $g\in G$; per dimostrare la proposizione definisco l'isomorfismo
	\begin{align*}
	\phi:\mathbb{Z}_n&\longrightarrow G\\
	k &\longmapsto g^k
	\end{align*}
	\begin{itemize}
		\item[(ben definita)] Siano $[a]=[b]\in\mathbb{Z}_n$ la stessa classe (con due rappresentanti diversi), per due certi $a,b\in\mathbb{Z}$; devo mostrare che la funzione non dipende dal rappresentante della classe. Notiamo che se $a$ e $b$ sono nella stessa classe allora $n$ divide la loro differenza. Questo scritto tramite la divisione euclidea (ad esempio) diventa $a-b=kn$, cioè $a=kn+b$, per un certo $k\in\mathbb{Z}$.
		\begin{align*}
		\phi([a])&=g^{[a]}=g^{[kn+b]}=g^{[kn]}g^{[b]}=1\cdot g^{[b]}\\
		&=g^{[b]}=\phi([b])
		\end{align*}
		Si noti l'utilizzo della struttura ciclica di $G$.
		\item[(omomorfismo)] Banalmente vero.
		\item[(iniettiva)] Vediamo che $\ker \phi = \left\{0\right\}$: se $x \in \ker \phi$ allora $\phi(x) = g^x = 1$, quindi $n \mid x$. Dev'essere che $x \equiv 0 \Mod n$, ovvero $x \equiv 0 \in \mathbb{Z}_n$. 
		\item[(suriettiva)] Per un fatto noto di algebra una funzione iniettiva tra insiemi finiti della stessa dimensione è anche suriettiva, ed i nostri insiemi sono ciclici.
	\end{itemize}
	Avendo trovato un isomorfismo abbiamo ottenuto la tesi.
\end{proof}
\begin{definizione}
	L'insieme di interi $\left\{a_1,\dots,a_n\right\}$ è un \textbf{sistema completo di residui modulo $n$} (oppure ancora \textbf{di rappresentanti di $\mathbb{Z}_n$}) se
	\begin{enumerate}
		\item $i\neq j \implies a_i\neq a_j \Mod(n)$%trovare il neq con 3 trattini orizzontali
		\item per ogni $a$ in $\mathbb{Z}_n$ esiste un indice $i$ tale che $a\equiv a_i \Mod(n)$
	\end{enumerate}
Il \textbf{sistema canonico di rappresentanti} di $\mathbb{Z}_n$ è
\begin{equation*}
1,2,\dots,n-1
\end{equation*}
\end{definizione}
\begin{proposizione}
	Siano $m,n$ in $\mathbb{Z}$ tali che $m$ divide $n$, allora la funzione
	\begin{align*}
	\phi:\mathbb{Z}_n&\longrightarrow\mathbb{Z}_m\\
	[a]_n&\longmapsto[a]_m
	\end{align*}
	è un omomorfismo suriettivo.
\end{proposizione}
\begin{proof}\
	\begin{itemize}
		\item[(ben definita)] Chiaramente se $[a]_n=[b]_n$ allora $n$ divide $a-b$, e quindi anche $m$ li divide; sono nella stessa classe anche rispetto ad $m$.
		\item[(omomorfismo)] Ovvio per struttura delle classi.
		\item[(suriettiva)] Ogni $[x]_m$ ha una controimmagine, $[x]_n$. Poiché $m$ divide $n$ allora è necessariamente più piccolo, quindi non posso \enquote{perdere classi}, o in altre parole non ho una riduzione dei rappresentanti. 
	\end{itemize}
\end{proof}
\begin{proposizione}
	Sia $n=ab$ con $a,b\in\mathbb{Z}$ coprimi, allora 
	\begin{equation*}
	(\mathbb{Z}_n,+)\simeq(\mathbb{Z}_a,+)\times(\mathbb{Z}_b,+)
	\end{equation*}
\end{proposizione}
\begin{proof}
	Definiamo le funzioni
	\begin{equation*}
		\begin{array}{rll}
			\phi \; \colon & \mathbb{Z}_n & \longrightarrow\mathbb{Z}_a \times \mathbb{Z}_b\\
			& [x]_n & \longmapsto\left([x]_a,[x]_b\right) \\
			 & & \\
			\phi_a \; \colon & \mathbb{Z}_n & \longrightarrow \mathbb{Z}_a \\
			& [x]_n & \longmapsto[x]_a\\
			 & & \\
			\phi_b \; \colon & \mathbb{Z}_n & \longrightarrow\mathbb{Z}_b \\
			& [x]_n & \longmapsto[x]_b
		\end{array}
	\end{equation*}
	Osserviamo subito che $\phi([x]_n)=\left(\phi_a([x]_n),\phi_b([x]_n)\right)$:
	\begin{equation*}
	%https://tikzcd.yichuanshen.de/#N4Igdg9gJgpgziAXAbVABwnAlgFyxMJZABgBoBGAXVJADcBDAGwFcYkQAdDgW3pwAsARoOAAtAL4B9QuNLpMufIRTlSxanSat2XXgOFip9ELPnY8BIqoBMGhizaJOPPkJETJgk3JAZzSomsKOy1HZz03Q0l6LjxueF1XAw8vcQ0YKABzeCJQADMAJwhuJDIQHAgkVRBGekEYRgAFBQtlECwwbFgQGnttJy40fixo73yiksQyiqQgmrqG5v9LJw6utl7QnQ4hkdSfQuLSmhnEAGYaWvqmloDVzqxuzYdt3bGQQ8nq04vy+ixGOx+BAIABrd6fWYnSrnE7-QFOYFgkyUcRAA
	\begin{tikzcd}
	& \mathbb{Z}_a \arrow[rd, hook] &                                \\
	\mathbb{Z}_n \arrow[ru, "\phi_a" description] \arrow[rd, "\phi_b" description] \arrow[rr, "\phi" description] &                               & \mathbb{Z}_a\times\mathbb{Z}_b \\
	& \mathbb{Z}_b \arrow[ru, hook] &                               
	\end{tikzcd}
	\end{equation*}
	quindi è un omomorfismo per quanto visto.
	\begin{itemize}
		\item[(iniettiva)] Studiamo il suo nucleo.
		\begin{align*}
		\ker(\phi)&=\left\{[x]_n\in\mathbb{Z}_n \, \mid \, \phi([x]_n)=\left([0]_a,[0]_b\right)\right\}\\
		&=\left\{[x]_n\in\mathbb{Z}_n \, \mid \, \phi([x]_n)=\big(\phi_a([0]_n),\phi_b([0]_n)\big)\right\}\\
		&=\left\{[x]_n\in\mathbb{Z}_n \, \mid \, \phi_a([x]_n)=0,\phi_b([x]_n)=0\right\}\\
		&=\left\{[0]_n\in\mathbb{Z}_n\right\}
		\end{align*}
		L'ultimo passaggio in particolare vale perché $a$ e $b$ sono coprimi, quindi il loro minimo comune multiplo è $ab=n$.
		\item[(suriettiva)] La cardinalità del gruppo ciclico $\mathbb{Z}_n$ è $n=ab$, che è il prodotto delle cardinalità di $\mathbb{Z}_a$ e $\mathbb{Z}_b$, concludiamo allo stesso modo di prima. Si noti che $\mathbb{Z}_a\times \mathbb{Z}_b$ è ciclico se $a$ e $b$ sono coprimi.
	\end{itemize}
\end{proof}
\paragraph{Struttura moltiplicativa}
\begin{proposizione}
	In $\left(\mathbb{Z}_n,+,\cdot\right)$ un elemento non nullo è invertibile rispetto al prodotto. Questo equivale a dire che per ogni $a$ in $\mathbb{Z}_n$ vale una delle seguenti:
	\begin{enumerate}
		\item $a$ ed $n$ sono coprimi
		\item $a$ è un divisore dello zero
	\end{enumerate}
\end{proposizione}
\begin{proof}
	$a$ ed $n$ sono coprimi equivale a dire che (per Bezout) esistono $x,y\in\mathbb{Z}$ con $ax+yn=1$, che equivale a dire
	\begin{equation*}
	a-x\equiv1\Mod(n)
	\end{equation*}
	Invece $\MCD(a,b)=d$ equivale a dire $a=dk_a$, $n=dk_n$ per $k_a,k_b\in\mathbb{Z}$ e $k_n\neq0$. Dunque 
	\begin{equation*}
	ak_n=(dk_a)k_n=k_an\equiv0\Mod(n)
	\end{equation*}
\end{proof}
\begin{osservazione}[Gruppo delle unità]
	Notiamo che $\left(\mathbb{Z}_n^*,\cdot\right)$ è un gruppo moltiplicativo di $\mathbb{Z}_n$, detto anche \textbf{gruppo delle unità modulo $n$}. \\ Diciamo $\varphi$ la funzione di Eulero, che abbiamo già incontrato.
	\begin{align*}
	\varphi: \mathbb{N}&\longrightarrow \mathbb{N}\\
	n &\longmapsto |\mathbb{Z}_n^*|
	\end{align*}
	Allora possiamo dire per $a$ e $b$ coprimi ed $n=ab$
	\begin{align*}
	\phi:\mathbb{Z}^*_n&\longrightarrow\mathbb{Z}^*_a\times \mathbb{Z}^*_b\\
	[x]_n&\longmapsto\left([x]_a,[x]_b\right)
	\end{align*}
	è un isomorfismo.
	\begin{itemize}
		\item[(ben definita)] Ovvio, basta riadattare la dimostrazione già vista.
		\item[(omomorfismo)] 
		\begin{align*}
		\phi\left([x]_n[y]_n\right)&=\phi\left([xy]_n\right)=\left([xy]_a,[xy]_b\right)\\
		&=\left([x]_a[y]_a,[x]_a[y]_a\right)=\phi\left([x]_n\right)\phi\left([y]_n\right)
		\end{align*}
		L'ultima uguaglianza andrebbe formalizzata meglio ripercorrendo i calcoli al contrario. \\
		\item[(iniettiva)] Il nucleo di $\phi$ sono gli elementi mandati nell'elemento neutro in arrivo, quindi $1$.\footnote{stiamo usando una notazione moltiplicativa per i gruppi; in poche parole, solitamente scriviamo l'operazione come $+$ e di conseguenza il neutro come $0$ ma è possibile usare la moltiplicazione (ed in quel caso l'elemento neutro sarà ovviamente 0).}
		Vediamo un fatto generale per la nostra situazione, $[x]_n$ è invertibile se e solo se lo sono $\left([x]_a\right)$ ed $\left([x]_b\right)$. Infatti se è invertibile si ha $[z]_n=[x]^{-1}_n$, quindi
		\begin{align*}
		\left([z]_a[x]_a,[z]_b[x]_b\right)&=\left([z]_a,[z]_b\right)\left([x]_a,[x]_b\right)\\
		&=\phi([z]_n)\phi([x]_n)\\
		&=\phi([1]_n)=\left([1]_a,[1]_b\right)
		\end{align*}
		Se invece sono invertibili $\left([x]_a\right)$ ed $\left([x]_b\right)$ esistono $[z]_a$ e $[z]_b$ loro inversi, quindi basta sfruttare il fatto che le funzioni $\phi_a$ e $\phi_b$ siano omomorfismi. \\ Otteniamo subito che la funzione è iniettiva.
		\item[(suriettiva)] Come le altre volte.
	\end{itemize}
\end{osservazione}
\begin{proposizione}\
	\begin{enumerate}
		\item La funzione di Eulero è una funzione aritmetica moltiplicativa.
		\item Per $p$ primo $\phi(p)=p-1$.
		\item $\phi(p^e)=p^{e-1}(p-1)$ per $p$ primo.
	\end{enumerate}
	Allora nota la fattorizzazione di un $n$ intero posso calcolare il valore di $\phi(n)$; il problema sarà proprio fattorizzare $n$.
\end{proposizione}
\begin{proof}\
	\begin{enumerate}
		\item Segue da quanto appena visto, per $n=ab$ con $a,b$ coprimi
		\begin{align*}
		\phi(ab)&=\phi(n)=|\mathbb{Z}_n^*|=|\mathbb{Z}_a^*\times\mathbb{Z}_b^*|\\
		&=|\mathbb{Z}_a^*|\cdot|\mathbb{Z}_b^*|=\phi(a)\phi(b)
		\end{align*}
		\item Ovvio per definizione di funzione di Eulero.
		\item Dato $a\in\mathbb{N}$ ed $a\in[1,p^e]$ si ha che
		\begin{align*}
		\MCD(a,p^e)=1&\iff \text{$p$ non divide $a$}\\
		\MCD(a,p^e)\neq1&\iff \text{$p$ divide $a$}
		\end{align*}
		quindi $\phi(p^e)$ è
		\begin{align*}
		p^e-\left|\left\{r\in\mathbb{N} \, \mid \, 1\leq r\leq p^e \ \text{e $p$ divide $r$}\right\}\right|
		\end{align*}
		Notando che l'insieme a destra è quello dei multipli di $p$ (quindi $p,2p,3p,\dots,p^{e-1}p$), otteniamo che 
		\begin{equation*}
		\phi(p^e)=p^e-p^{e-1}=p^{e-1}(p-1)
		\end{equation*}
	\end{enumerate}
\end{proof}
\begin{teorema}[Piccolo teorema di Fermat]
	Per ogni $a\in\mathbb{Z}_p^*$ con $p$ primo, vale la seguente formula.
	\begin{equation*}
	a^{p-1}\equiv1\Mod(p)
	\end{equation*}
\end{teorema}
\begin{proof}[Dimostrazione \enquote{elementare}]
	Dimostriamolo per induzione su $a$.
	\begin{enumerate}
		\item[($a=1$)]La tesi  $1^{p-1}=1\equiv1\Mod(p)$, che è banalmente vera. 
		\item[($a+1$)] Riscriviamo facilmente l'ipotesi induttiva.
		\begin{equation*}
		a^{p-1}\equiv1\Mod(p)\iff a^p\equiv a\Mod(p)
		\end{equation*}
		Dimostriamo il passo $a+1$-esimo. Il fatto che $(a+1)^p\equiv(a^p+1)\Mod(p)$ è ben noto dall'algebra, basta scrivere la formula di Newton per l'espansione della potenza $p$. Questo implica subito la tesi sfruttando l'ipotesi induttiva:
		\begin{align*}
		(a)^{p}\equiv(a)\Mod(p) \ \ \ \ \ \ &\text{ipotesi induttiva}\\
		(a+1)^p\equiv(a^p+1)\Mod(p) \ \ \ \ \ \ &\text{fatto noto}\\
		\end{align*}
		\begin{equation*}
		(a+1)^{p-1}\equiv (a^p+1)\mod(p)\equiv (a+1)\mod(p)
		\end{equation*}
	\end{enumerate}
\end{proof}
\begin{proof}[Dimostrazione \enquote{avanzata}] 
	Si può dimostrare anche come banale conseguenza delle proprietà della funzione di Eulero e della teoria che abbiamo appena visto. \\ Non lo faremo, ma è possibile. 
	%todo: dimostrazione combinatorica, vedi materiali extra
\end{proof}
\begin{teorema}[Teorema di Eulero]
	Per ogni $a\in\mathbb{Z}_n^*$ vale la seguente formula.
	\begin{equation*}
	a^{\phi(n)}\equiv 1 \Mod(n)
	\end{equation*}
\end{teorema}
\begin{proof}
	Ricordiamo che $|\mathbb{Z}^*_n| = \phi(n)$, da cui poiché $\langle a\rangle \le \mathbb{Z}^*_n$ dev'essere che l'ordine di $|\langle a\rangle| = o(a) \mid |\mathbb{Z}^*_n|$ dal teorema di Lagrange. Si conclude per definizione di ordine: $1^k = a^{o(a)k} = a^{\phi(n)}$ ovvero la tesi. 
\end{proof}




\subsection{Congruenze lineari e teorema cinese del resto}
Diciamo \textbf{congruenze lineari} equazioni del tipo
\begin{equation*}
ax\equiv b\Mod(n)
\end{equation*}
Notiamo alcune cose:
\begin{enumerate}
	\item Se $a$ è invertibile (in $\mathbb{Z}_n$), ovvero $a$ ed $n$ sono coprimi per caratterizzazione vista, notiamo subito che si risolve ponendo
	\begin{equation*}
	x\equiv a^{-1}b\Mod(n)
	\end{equation*}
	\item Se $a$ non è invertibile, ovvero $a$ ed $n$ hanno massimo comune divisore $d\neq1$, si risolve se e solo se $d$ divide $b$. In questo caso infatti si pone $a'=\frac{a}{d}$, $b'=\frac{b}{d}$, $n'=\frac{n}{d}$ (nota: $a'$ ed $n'$ sono coprimi).\\
	Questo perché il problema equivale a risolvere l'equazione diofantina 
	\begin{equation*}
	ax+ny=b
	\end{equation*}
	nelle incognite $x$ ed $y$, e se $d$ divide $b$ abbiamo che $d$ divide $ax+ny$.
	\begin{equation*}
	a'x+n'y=b' \iff a'x\equiv b'\Mod(n)
	\end{equation*}
\end{enumerate}
Poniamo invece di voler risolvere \textit{un sistema} di $k$ congruenze lineari.
\begin{equation*}
\begin{cases}
a_1x&\equiv b_1\Mod(n_1)\\
&\,\vdots\\
a_kx&\equiv b_k\Mod(n_k)
\end{cases}
\end{equation*}
Allora ogni equazione deve innanzi tutto essere risolvibile secondo il criterio appena studiato, 
\begin{equation*}
\MCD(a_i,n_i)\mid b_i \ \ \forall i \in \left\{1, \dots, k\right\}
\end{equation*}
ma stavolta non è sufficiente a garantire l'esistenza di una soluzione.
\begin{controesempio}
	Le condizioni non sono sufficienti a garantire l'esistenza di una soluzione. Si consideri il seguente sistema lineare 
	\begin{equation*}
	\begin{cases}
		x\equiv 2\Mod(4)\\
		x\equiv 1\Mod(2)
	\end{cases}
	\end{equation*}
	La prima ci dice che $x = 4k + 2$ e quindi ha come soluzioni numeri pari; la seconda invece dice che $x = 2k + 1$, ovvero ha soluzioni dispari. Un numero intero non può essere sia pari che dispari: la soluzione non esiste. \\ \\ Questo è evidentemente un caso comune; ogni equazione di per sé ha (sempre) un insieme di soluzioni e nulla mi vieta di mettere a sistema due equazioni che non abbiano alcuna soluzione in comune.
\end{controesempio}
\begin{osservazione}
	Consideriamo ad esempio il sistema di due equazioni
	\begin{equation*}
	\begin{cases}
	a_1x\equiv b_1\Mod(n_1)\\
	a_2x\equiv b_2\mod(n_2)
	\end{cases}
	\end{equation*}
	che sotto le condizioni viste può essere riscritto come 
	\begin{equation*}
	\begin{cases}
	x\equiv c_1\Mod(n_1)\\
	x\equiv c_2\Mod(n_2)
	\end{cases}
	\end{equation*}
	\begin{equation*}
	\begin{cases}
	x=c_1+kn_1'\\
	x=c_2+hn_2'
	\end{cases}
	\end{equation*}
	Otteniamo l'equazione nelle incognite $h,k$
	\begin{equation*}
	c_1-c_2=hn_2'-kn_1'
	\end{equation*}
	che per Bezout ha soluzione se e solo se $\MCD(n_1',n_2') \mid c_1-c_2$.
\end{osservazione}
\begin{teorema}[Teorema cinese del resto]
	Siano $m,n\in\mathbb{Z}$ coprimi ed $a,b\in\mathbb{Z}$. Allora il sistema 
	\begin{equation*}
	\begin{cases}
	x\equiv a\Mod(m)\\
	x\equiv b\Mod(n)
	\end{cases}
	\end{equation*}
	ha una ed una sola soluzione in modulo $mn$.
\end{teorema}
\begin{proof}
	Consideriamo la funzione 
	\begin{align*}
	\varphi:\mathbb{Z}&\longrightarrow\mathbb{Z}_m\times\mathbb{Z}_n\\
	x&\longmapsto\left([x]_m,[x]_n\right)
	\end{align*}
	Se è suriettiva ottengo subito che esiste un $x\in\mathbb{Z}$ tale che $\phi(x)=\left([a]_m,[b]_n\right)$, e questa è la soluzione del sistema.
	\begin{itemize}
		\item[(ben definita)] Già visto.
		\item[(omomorfismo)] Ovvio.
		\item[(iniettiva?)] Non è iniettiva, il nucleo è non banale.
		\begin{align*}
		\ker(\varphi)&=\left\{x\in\mathbb{Z} \, \middle| \, x\equiv 0 \Mod(n),x \equiv 0 \Mod(m)\right\}\\
		&=\left\{x\in\mathbb{Z} \, \middle| \, mn
		\mid x\right\}=(mn)\mathbb{Z}
		\end{align*}
		\item[(suriettiva)]
		La funzione è palesemente suriettiva per costruzione, tutto lo spazio di arrivo viene coperto.
	\end{itemize}
	Il nucleo è non banale, ma la mappa è ben definita e suriettiva; posso sfruttare il primo teorema di isomorfismo tra anelli ed ottengo l'isomorfismo
	\begin{equation*}
	\sfrac{\mathbb{Z}}{\ker(\varphi)}\simeq\mathbb{Z}_m\times\mathbb{Z}_n\simeq\mathbb{Z}_{mn}
	\end{equation*}
	Allora se $x$ è soluzione, tutte le soluzioni hanno forma 
	\begin{equation*}
	x+k(mn) \ \ \ \ k\in\mathbb{Z}
	\end{equation*}
	ovvero ho esistenza ed unicità di soluzione in $\mathbb{Z}_{mn}$. Questo perché le ipotesi del teorema mi mettono nella situazione di esistenza della soluzione che abbiamo appena studiato.
\end{proof}
\begin{proposizione}
	Sia data la fattorizzazione di $n$
	\begin{equation*}
	n=\prod_{i=1}^{n}{p_i}^{{e_i}}=\alpha\cdot\beta\cdot\ \dots \ \omega
	\end{equation*}
	dove $\alpha,\beta,\dots,\omega$ sono i fattori (i primi, esponente incluso). Allora 
	\begin{equation*}
	\mathbb{Z}_n\simeq\mathbb{Z}_{\alpha}\times\ \dots \ \times\mathbb{Z}_{\omega}
	\end{equation*}
	%inserire gli alpha è stato necessario per grane varie di latex con i pedici
	In particolare abbiamo anche che 
	\begin{equation*}
	\mathbb{Z}_n^*\simeq\mathbb{Z}_{\alpha}^*\times\ \dots \ \times\mathbb{Z}_{\omega}^*
	\end{equation*}
\end{proposizione}
\begin{proof}
	Si svolge considerando il morfismo
	\begin{align*}
	\varphi:\mathbb{Z}&\longrightarrow\mathbb{Z}_{\alpha}\times\ \dots \ \times\mathbb{Z}_{\omega}\\
	x&\longmapsto\left([x]_{\alpha},\dots,[x]_{\omega}\right)
	\end{align*}
	e dimostrando che è isomorfismo.
	%todo: proof lunga
\end{proof}
\begin{teorema}[Teorema cinese del resto, generale]
	Siano $n_1,\dots,n_k\in\mathbb{Z}$ due a due coprimi ed $a_1,\dots,a_k\in\mathbb{Z}$. Allora il sistema 
	\begin{equation*}
	\begin{cases}
	x&\equiv a_1\Mod(n_1)\\
	&\,\vdots\\
	x&\equiv a_k\Mod(n_k)
	\end{cases}
	\end{equation*}
	ha una ed una sola soluzione in modulo $mn$.
\end{teorema}




\subsection{La struttura ciclica di $\mathbb{Z}^*_p$}
\label{lezione7}
Ovviamente prenderemo come $p$ un primo di $\mathbb{Z}$.
\begin{proposizione}
	Sia $G=\left\{g_1,\dots,g_s\right\}$ un gruppo abeliano finito, sia $n$ il massimo degli ordini dei suoi elementi. \\ Allora l'ordine di ogni elemento divide $n$.\footnote{non è scontato che $n$ sia uguale ad $s$, anzi se il gruppo non è ciclico vale $n<s$.}
\end{proposizione}
\begin{proof}
	Sia $g\in G$ l'elemento di ordine $n$ massimo e sia $h\in G$ di ordine $m$ generico. Supponiamo per assurdo che $m$ non divida $n$.\\ \\
	Se $m$ ed $n$ fossero coprimi allora $gh\in G$ avrebbe ordine $mn$, ma questo è in contraddizione con la scelta di $n$. Allora non sono coprimi. Definiamo $\nu_p(x)$ come la valutazione $p$-adica di $x$, ovvero la massima potenza di $p$ che divide $x$; l'idea è avere un primo che divida entrambi con potenze diverse. Sia allora $p$ primo con $\nu_p(m)=e$, $\nu_p(n)=f$, evidentemente $0>e>f$. Consideriamo l'elemento $g^{pf}$, verifichiamo facilmente che ha ordine $\frac{n}{pf}$. 
	\begin{equation*}
	\left(g^{pf}\right)^{\sfrac{n}{pf}}=g^n=1
	\end{equation*}
	Analogamente $h^{\sfrac{m}{pe}}$ ha ordine $p^e$.
	\begin{equation*}
	\left(h^{\sfrac{m}{pe}}\right)^{p^e}=h^m=1
	\end{equation*}
	Siccome questi due ordini sono coprimi, $\left(g^{p}\right)^fh^{\sfrac{m}{p^e}}$ appartiene a $G$ ed ha ordine $\frac{n}{pf}p^e=p^{e-f}n>n$; questo contraddice di nuovo la scelta di $n$.
\end{proof}
\begin{teorema}
	Sia $f\in\mathbb{Z}_p[x]$ polinomio non costante e di grado $d$ (nella variabile $x$), allora ha al più $d$ radici in $\mathbb{Z}_p$.
\end{teorema}
\begin{proof}
	Iniziamo ricordando che per teorema di Ruffini $f(a)=0$ se e solo se $(x-a)$ divide $f$. Procediamo per induzione su $d$.
	\begin{itemize}
		\item[($d=1$)] Banale, sono nel caso $f(x)=ax+b$ per $a$ non-zero ed $a,b\in\mathbb{Z}_p$. \\ Ovviamente ho una sola radice: $-ba^{-1}$.
		\item[($d+1$)] Sia ora 
		\begin{equation*}
		f(x)=c_0+\ \dots\ +c_{d+1}x^{d+1}  \ \ \ \ \ \ \ \ c_1\in\mathbb{Z}_p,\ c_{d+1}\neq 0
		\end{equation*}
		Se non ha radici ottengo la tesi, altrimenti esiste un $a\in\mathbb{Z}_p$ tale che $f(a)=0$. In questo caso posso riscrivere $f$ come 
		\begin{equation*}
		f(x)=(x-a)g(x)
		\end{equation*}
		per un certo polinomio $g(x)\in\mathbb{Z}_p[x]$ di grado $d$, su cui posso usare l'ipotesi induttiva.
	\end{itemize}
\end{proof}
\begin{teorema}
	$\left(\mathbb{Z}_p^*,\cdot\right)$ è un gruppo ciclico.
\end{teorema}
\begin{proof}
	Sia $n$ il massimo degli ordini degli $i$ compresi tra $1$ e $p-1$, voglio mostrare che $n=p-1$; in questo modo otterrei che esiste almeno un elemento di ordine $p-1$ e da qui la tesi. \\ Sicuramente $n\leq p-1$ ed $n$ divide $p-1$, in particolare ogni elemento $a$ del gruppo ha ordine $m$ che divide $n$ e quindi
	\begin{equation*}
	a^m\equiv a^n\equiv 1\Mod(p) 
	\end{equation*}
	Questo vuol dire che $x^n-1\in\mathbb{Z}_p[x]$ ha almeno $p-1$ soluzioni, quindi $p-1\leq n$.
\end{proof}
\begin{osservazione}
	Anche se sappiamo che $\left(\mathbb{Z}_p^*,\cdot\right)$ è ciclico non sempre è facile trovare un suo generatore. 
	\begin{equation*}
	\left(\mathbb{Z}_p^*,\cdot\right)\simeq\left(\mathbb{Z}_{p-1},+\right)
	%todo: controllare se è giusta la formula
	\end{equation*}
\end{osservazione}

\subsection{La struttura ciclica di $\mathbb{Z}_n^*$}
Abbiamo visto che $\left(\mathbb{Z}_n,+\right)$ è ciclico per ogni $n$ e $\left(\mathbb{Z}_p^*,\cdot\right)$ è ciclico per $p$ primo. \\ Non per tutti gli $n$ $\left(\mathbb{Z}_n^*,\cdot\right)$ è ciclico; come possiamo caratterizzare gli $n$ per cui vale?
\begin{definizione}
	Un $a\in\mathbb{Z}_n^*$ si dice \textbf{radice primitiva modulo $n$} se ha ordine $\varphi(n)$
	, ovvero se è un generatore di $\mathbb{Z}_n^*$.
\end{definizione}
\begin{esempio}
	Osserviamo alcuni esempi di radici primitive.
	\begin{itemize}
		\item $\mathbb{Z}_2^*=\left\{1\right\}$, è ciclico
		\item $\mathbb{Z}_4^*=\left\{1,3\right\}$, è ciclico; oltretutto $\left(\mathbb{Z}_4^*,\cdot\right)\simeq\left(\mathbb{Z}_{2},+\right)$
	\end{itemize}
\end{esempio}
\begin{proposizione}
	Per $n$ pari e per $a\in\mathbb{Z}_n^*$ dispari si ha 
	\begin{equation*}
	a^{2^{k-2}}\equiv1\Mod\left(2^k\right)
	\end{equation*} 
\end{proposizione}
\begin{proof}
	Procediamolo per induzione su $k$, l'ipotesi induttiva è 
	\begin{equation*}
	a^{2^{k-2}}\equiv1\Mod2^k \iff a^{\displaystyle2^{k-2}}=1+t2^k \ \ \ \ \ t \in \mathbb{Z}
	\end{equation*}
	Per provarlo bastano pochi calcoli.
	\begin{align*}
	a^{2^{(k+1)-2}}&=a^{2^{k-2}\cdot2}=\left(a^{2^{k-2}}\right)^2=\left(1+t2^k\right)^2\\
	&= 1+t2^{k+1}+t^22^{2k}\equiv1\Mod2^{k+1}
	\end{align*}
\end{proof}
\begin{esempio}\
	\begin{itemize}
		\item $\mathbb{Z}_8^*=\left\{1,3,5,7\right\}$, $1^2\equiv3^2\equiv5^2\equiv7^2\equiv1\Mod8$
		\item posso ripetere la stessa procedura per ogni $n=2^k$ pari con $k>2$
	\end{itemize}
\end{esempio}
\begin{teorema}
	$\left(\mathbb{Z}_{2^k}^*,\cdot\right)$ è ciclico se e solo se $k=1,2$.
\end{teorema}
\begin{teorema}
	Sia $g$ generatore di $\mathbb{Z}_p^*$, allora $g$ o $g+p$ è un generatore di $\mathbb{Z}_{p^2}^*$.
\end{teorema}
\begin{proof}
	Sia $m$ l'ordine di $g$ in $\mathbb{Z}_{p^2}^*$, abbiamo che
	\begin{enumerate}
		\item $p-1$ divide $m$
		\item $m$ divide $p(p-1)$
	\end{enumerate}
	e quindi $m=p-1$ oppure $m=p(p-1)$ per unicità della fattorizzazione.
	\begin{itemize}
		\item[($m=p-1$)] In maniera analoga dimostriamo che dato $n$ ordine di $g+p$ in $\mathbb{Z}_{p^2}^*$ si ha $n=p-1$ oppure $n=p(p-1)$. Se fosse $n=p-1$
		\begin{align*}
		(g+p)^{p-1}&\equiv g^{p-1}+(p-1)pg^{p-2}\Mod p^2\\ 
		&\equiv 1-pg^{p-2}\Mod p^2 \nequiv 1\Mod p^2
		\end{align*}
		e quindi $p^2$ non divide $pg^{p-2}$, $n\neq p-1$.
		\item[($m=p(p-1)$)] Caso ovvio, $g$ genera $\mathbb{Z}_{p^2}^*$.
	\end{itemize}
\end{proof}
\begin{teorema}
	Un generatore di $\mathbb{Z}_{p^2}^*$ è un generatore di $\mathbb{Z}_{p^e}^*$, $e\geq2$.
\end{teorema}
\begin{proof}
	Siano $e\geq2$ e $g$ generatore di $\mathbb{Z}_{p^e}^*$, voglio mostrare che è generatore anche di $\mathbb{Z}_{p^{e+1}}^*$. \\ \\ Sia $d$ l'ordine di $g$ in $\mathbb{Z}_{p^{e+1}}^*$, come prima notiamo che 
	\begin{itemize}
		\item $p^{e-1}(p-1)$ divide $d$
		\item $d$ divide $p^e(p-1)$
		\item $p^e(p-1)=p^{e-1}p(p-1)$
	\end{itemize}
Allora $d=p^{e-1}(p-1)$ oppure $d=p^e(p-1)$. Vogliamo vedere che $g^{p^{e-1}(p-1)}$ non equivale ad $1\Mod p^{e+1}$. \\ Sappiamo che:
\begin{itemize}
	\item $g^{p^{e-2}(p-1)}$ non equivale ad $1\Mod p^e$
	\item $g^{p^{e-2}(p-1)}$ equivale ad $1\Mod p^{e-1}$
\end{itemize}
quindi $g^{p^{e-2}(p-1)}=1+kp^{e-1}$ e $p$ non divide $k$.
\begin{align*}
g^{p^{e-1}(p-1)}&=\left(1+kp^{e-1}\right)^p\equiv 1+kp^e+\frac{1}{2}k^2p^{2e-1}(p-1)\Mod p^{e+1}\\
&\equiv 1+kp^e\Mod p^{e+1}\nequiv1\Mod p^{e+1}
\end{align*}
Nell'ultimo passaggio ho usato il fatto che $p$ non divide $k$.
\end{proof}
\begin{teorema}
	$\left(\mathbb{Z}_{2p^k}^*,\cdot\right)$ è ciclico.
\end{teorema}
\begin{proof}
	Scelgo $p$ dispari siccome ho già trattato il caso $p$ primo pari ($p=2$). Dato che $2$ e $p^k$ sono ciclici, 
	\begin{equation*}
	\mathbb{Z}_{2p^k}^*\simeq\mathbb{Z}_{2}^*\times\mathbb{Z}_{p^k}^*\times\mathbb{Z}_{p^k}^*
	\end{equation*}
	questo perché $\mathbb{Z}_{2}^*$ è il gruppo banale.
\end{proof}
\begin{teorema}
	$\left(\mathbb{Z}_{n}^*,\cdot\right)$ è un gruppo ciclico se e solo se $n=2,4,p^k,2p^k$ per $p$ primo dispari e $k>0$.
\end{teorema}
\begin{proof}
	Abbiamo già visto che per tali valori di $n$ ottengo un gruppo ciclico, ma ce ne possono essere altri?\\ \\
	Sia $n$ generico, costruisco la sua fattorizzazione $n=p_1^{e_1}\dots p_r^{e_r}$. Come sappiamo
	\begin{equation*}
	\mathbb{Z}_n^*\simeq\mathbb{Z}_{p_1^{e_1}}^*\times\dots\times\mathbb{Z}_{p_r^{e_r}}^*
	\end{equation*}
	e possiamo sfruttare il fatto che $\left|\mathbb{Z}_{p_j^{e_j}}^*\right|=\varphi\left(p_j^{e_j}\right)$, sempre pari tranne per $p_i=2$ ed $e_i=1$. Ci basta mostrare che se $G$ ed $H$ sono gruppi ciclici di ordine pari, allora $G\times H$ non è ciclico. Sia $2a$ l'ordine di $G$, $2b$ quello di $H$. \\ Sia $(x,y)\in G\times H$, allora
	\begin{equation*}
	(x,y)^{2ab}=(x^{2ab},y^{2ab})=(1,1)
	\end{equation*}
	ma l'ordine di $G\times H$ è $4ab$. Non esistono elementi di $G\times H$ che abbiano come ordine il suo.
\end{proof}








%\section{Residui quadratici}
