\begin{esercizio}
	Quanti zeri terminali ha $2020!$?
	
	Essenzialmente dobbiamo trovare $n$ massimo per cui $10^n \mid 2020!$. Per il teorema fondamentale dell'aritmetica segue che 
	$5^n \mid 2020!$ e $2^n \mid 2020!$. Calcoliamo la potenza massima di $5$ tale per cui divida $2020!$. 
	Allora la formula è ovviamente 
	\begin{equation*}
		n = \left\lfloor \frac{2020}{5} \right\rfloor + \left\lfloor \frac{2020}{5^2} \right\rfloor + \left\lfloor \frac{2020}{5^3} \right\rfloor + \left\lfloor \frac{2020}{5^4} \right\rfloor = 404 + 80 + 16 + 3 = 503
	\end{equation*}
    da cui $503$ è il numero di $0$ terminali.
\end{esercizio}

\begin{esercizio}
	Dimostrare che per ogni $n > 0$, $682 \mid (n^{31} - n)$.
	
	Basta usare il teorema cinese dei resti rispetto a $n^{31} - n \equiv 0 \mod 682$.
\end{esercizio}

\begin{esercizio}
	Se $x^4 - x$ e $x^3 - x$ sono interi per un qualche $x \in \mathbb{R}$, allora $x$ è intero.
	
	Dato che $\gcd(x^4 - x, x^3 - x) = x^2 - x \in \mathbb{Z}$.
	Ma 
	\begin{equation*}
		\frac{x^3 - x}{x^2 - x} = 1+x \in \mathbb{Z}
	\end{equation*}
	dato che è una divisione di interi tale che uno divide l'altro $1+x$ è intero e dunque $x \in \mathbb{Z}$.
\end{esercizio}

\begin{esercizio}
	Dimostrare che se $p$, $p^2 + 2$ sono primi allora lo è anche $p^3 +2$.\\
	
	Questo esercizio rivela l'utilità di provare a fare i calcoli. 
	\begin{enumerate}
		\item Infatti non può essere che $p = 2$, 
		guardando modulo $2$ si vede che 
		$p^2 + 2 \equiv 0 \mod 2$, ma $p^2 + 2 > 2$ quindi non è mai primo.
		\item Se si guarda invece modulo $3$ si ottiene che
		$p^2 + 2 \equiv 0 + 2 \equiv 2 \mod 3$ e quindi ammette 
		la soluzione $3, 11, 29$.
		\item Se invece è un numero primo $p > 3$ allora sarà tale che $p \equiv \pm 1\mod 3$. 
		Allora $p^2 + 2 \equiv 1 + 2 \equiv 0 \mod 3$ e 
		dunque $p^2 + 2$ non è mai primo.  
	\end{enumerate}	 
\end{esercizio}


\begin{esercizio}
	Se $A$ anello noetheriano allora $A/I$ è noetheriano dove $I$ ideale di $A$.\\
	
	Sia $\pi \colon A \to A/I$ l'omomorfismo di proiezione al quoziente.
	Allora sappiamo che esiste una corrispondenza tra gli ideali $J \subset A$ tali
	che $I \subset J$ e gli ideali $\pi(J) \subset A/I$. 
	Quindi sia $K$ ideale di $A/I$ allora esiste $J$ ideale di $A$ tale che 
	$\pi(J) = K$. A questo punto osserviamo che $K$ dev'essere finitamente generato:
	sia $a \in K$ allora $a \in \pi(J)$ quindi esiste $b \in J$ tale che $\pi(b) =a$
	ma sappiamo che $J = (w_1, \dots, w_n)$ ideale finitamente generato poiché $A$ noetheriano. Pertanto 
	\begin{equation*}
		\pi(b) = \pi(a_1 w_1 + \dots + a_n w_n) = \pi(a_1) \pi(w_1) + \dots + 
					\pi(a_n)\pi(w_n) = a
	\end{equation*}     
 	pertanto basta prendere $\pi(w_1), \dots, \pi(w_n)$ come 
 	generatori di $K$. 
\end{esercizio}

\begin{esercizio}
	Se $A$ anello noetheriano e $B$ è un anello e un $A$-modulo finitamente generato\footnote{In questo caso $B$ diventa una $A$ algebra.} allora $B$ è noetheriano.\\
	
	Ricordiamo la definizione di finitamente generato:
	Un $A$-modulo $B$ si dice finitamente generato se esistono 
	degli elementi $b_1, \dots, b_n$ tali che per ogni $b \in B$ è possibile trovare $b = f(a_1)b_1 + \dots + f(a_n)b_n$, dove $f \colon A \to B$ è l'omomorfismo che
	descrive l'azione prodotto di $a\cdot b$ con $a\in A$, $b\in B$.\\
	
	Allora possiamo vedere che è equivalente a dire che esiste
	un omomorfismo suriettivo $\varphi \colon A^r \to B$ per 
	qualche $r > 0$. Ma per il teorema di primo isomorfismo
	otteniamo che $A^r/\ker \varphi \cong B$. La noetherianità 
	si preserva passando al quoziente (si veda es. 
	precedente). Inoltre viene preservata per isomorfismi di
	anelli. Per cui $B$ risulta essere noetheriano. 
\end{esercizio}

\begin{esercizio}
	Trovare un metodo alternativo per dimostrare che $\mathbb{Z}[\sqrt{d}i]$ per ogni $d$ square-free 
	è un dominio con fattorizzazione (non per forza di cose unica).\\
	
	L'idea qui è di usare gli esercizi precedenti per osservare che $\mathbb{Z}[\sqrt{d}i]$ non è 
	nient'altro che è un $\mathbb{Z}$-algebra finitamente generata come $\mathbb{Z}$-modulo e quindi
	è noetheriano se e solo se $\mathbb{Z}$ lo è. Ovviamente $\mathbb{Z}$ è noetheriano.
	In particolare se un anello è noetheriano è anche principalmente noetheriano, da cui discende 
	la tesi dell'esistenza della fattorizzazione.
\end{esercizio}

\begin{esercizio}

	Sia $A$ anello integralmente chiuso sul suo campo delle frazioni $K$. Sia $f(x) \in A[x]$ monico. 
	Dimostrare che se $f$ è riducibile in $K[x]$ allora $f$ riducibile in $A[x]$\\
	
	Siccome $f$ è riducibile in $K[x]$ vuol dire che esiste $\alpha \in K$ tale che $(x-\alpha)g(x) = f(\alpha) = 0$. Ma $\alpha \in A$ 
	dato che è integralmente chiuso, pertanto $f(x) = (x-\alpha)g(x)$ per qualche $g(x) \in A[x]$.\\ 
	
	In particolare risulta immediato (per contrapposizione), 
	che se $f \in A[x]$ monico e $f$ irriducibile in $A[x]$
	allora è irriducibile anche in $K[x]$. 
\end{esercizio}

\begin{esercizio}
	Trovare per quali $p$ primi ammette soluzioni non banali la seguente equazione
	\begin{equation*}
		x^2 - 5y^2 \equiv 0 \mod p
	\end{equation*}
	
	È ovvio che ammette soluzioni se e solo se $a^2 \equiv 5 \mod p$ dato che i 
	quadrati sono un gruppo rispetto al prodotto in un campo. Pertanto studiamo
	attraverso i potenti mezzi della reciprocità quadratica quando $5$ risulta essere 
	un quadrato. Per reciprocità quadratica
	\begin{equation*}
		 \left(\frac{5}{p}\right) \left(\frac{p}{5}\right) = (-1)^{p-1}
	\end{equation*}
	dunque abbiamo due casi 
	\begin{equation*}
		\begin{cases}
			\left(\frac{p}{5}\right) = 1 \iff p \equiv  1, 4 \mod 5 \\
			\left(\frac{p}{5}\right) = -1 \iff p \equiv 2, 3 \mod 5 \\	
		\end{cases}
	\end{equation*}
	dato che $p$ è primo la classe di resto $0$ non può essere considerata.
	\begin{enumerate}
		\item Sia $p \equiv 1 \mod 5$ allora 
			\begin{equation*}
				\left(\frac{5}{p}\right) = (-1)^{p-1}
			\end{equation*}
			a questo punto c'è solo un caso $p \equiv 1 \mod 2$. Allora otteniamo 
			che l'unico modo per cui  $(-1)^{p-1}= 1$ dev'essere per
			$p \equiv 1 \mod 2$. Attraverso il TCR otteniamo la soluzione
			\begin{equation*}
				\begin{cases}
					p \equiv 1 \mod 5 \\
					p \equiv 1 \mod 2
				\end{cases}
			\end{equation*}
			ovvero $p = 10n + 1$.
		\item Sia $p \equiv 4 \mod 5$ allora 
			\begin{equation*}
				\left(\frac{5}{p}\right) = (-1)^{p-1}
			\end{equation*}
			a questo punto c'è solo un caso per $p \equiv 1 \mod 2$. Allora otteniamo 
			che l'unico modo per cui  $(-1)^{p-1}= 1$ dev'essere per
			$p \equiv 1 \mod 2$. Attraverso il TCR otteniamo la soluzione
			\begin{equation*}
				\begin{cases}
					p \equiv 4 \mod 5 \\
					p \equiv 1 \mod 2
				\end{cases}
			\end{equation*}
			ovvero $p = 10n + 9$.
		\item Sia $p \equiv 2 \mod 5$ allora 
			\begin{equation*}
				\left(\frac{5}{p}\right) = -(-1)^{\frac{p-1}{2}}
			\end{equation*}
			a questo punto vediamo che $p \equiv 1 \mod 2$ e dunque non
			abbiamo soluzioni.
		\item Sia $p \equiv 3 \mod 5$ allora 
			\begin{equation*}
				\left(\frac{5}{p}\right) = -(-1)^{\frac{p-1}{2}}
			\end{equation*}
			a questo punto vediamo che $p \equiv 1 \mod 2$ e dunque non
			abbiamo soluzioni.
	\end{enumerate}
	Per cui alcuni esempi sono per $p = 11, 19, \dots$.
\end{esercizio}

