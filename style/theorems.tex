% [OPERATORI] --------------------------------

% nuovi operatori matematici
\DeclareMathOperator{\MCD}{\mathrm{MCD}}
\DeclareMathOperator{\dv}{div}
\DeclareMathOperator{\tr}{tr}
\DeclareMathOperator{\rank}{rank}
\DeclareMathOperator{\rt}{rot}
\DeclareMathOperator{\Log}{Log}
\DeclareMathOperator{\im}{\mathfrak{Im}}
\DeclareMathOperator{\re}{\mathfrak{Re}}
\DeclareMathOperator{\ind}{Ind}
\DeclareMathOperator{\supp}{supp}
\DeclareMathOperator{\res}{Res}
\DeclareMathOperator{\ord}{ord}


% [COMANDI] --------------------------------

% nuovi comandi per comodità di scrittura
\newcommand{\Mod}[1]{\ \mathrm{mod}\ #1}
\newcommand{\leg}[2]{\left(\frac{#1}{#2}\right)}


% [THEOREMSTYLE] --------------------------------

% stile dei teoremi
\theoremstyle{definition}

%todo: sistemare la numerazione in modo che proposizioni e teoremi condividano la numerazione? un po' confuso così
\newtheorem{definizione}{Definizione}[chapter]
\newtheorem {osservazione}{Osservazione}[chapter]
\newtheorem{proposizione}{Proposizione}[chapter]
\newtheorem{esempio}{Esempio}[chapter]
\newtheorem{teorema}{Teorema}[chapter]
\newtheorem{corollario}{Corollario}[teorema]
\newtheorem{controesempio}{Controesempio}[teorema]
\newtheorem{esercizio}{Esercizio}[chapter]
\newtheorem{algoritmo}{Algoritmo}[chapter]
\newtheorem{lemma}{Lemma}[chapter]
