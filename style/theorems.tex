% [OPERATORI] --------------------------------

% nuovi operatori matematici
\DeclareMathOperator{\dv}{div}
\DeclareMathOperator{\tr}{tr}
\DeclareMathOperator{\rank}{rank}
\DeclareMathOperator{\rt}{rot}
\DeclareMathOperator{\Log}{Log}
\DeclareMathOperator{\im}{\mathfrak{Im}}
\DeclareMathOperator{\re}{\mathfrak{Re}}
\DeclareMathOperator{\ind}{Ind}
\DeclareMathOperator{\supp}{supp}
\DeclareMathOperator{\res}{Res}
\DeclareMathOperator{\ord}{ord}


% [COMANDI] --------------------------------

% nuovi comandi per comodità di scrittura
\newcommand{\xx}{\textbf{x}}
\newcommand{\xxp}{\textbf{x'}}
\newcommand{\nx}{\nabla_{\textbf{x}}}
\newcommand{\rx}{\rt_\xx}
\newcommand{\vx}{v_E(t,\xx)}
\newcommand{\ve}{v_E}
\newcommand{\dx}{\dv_\xx}
\newcommand{\dphi}{\Delta_\xx \varphi(\xx)}
\newcommand{\parxx}{\frac{\partial ^2}{\partial x^2}}
\newcommand{\parxy}{\frac{\partial ^2}{\partial x \partial y}}
\newcommand{\paryy}{\frac{\partial ^2}{\partial y^2}}


% [THEOREMSTYLE] --------------------------------

% stile dei teoremi
\theoremstyle{definition}

\newtheorem{definizione}{Definizione}
\newtheorem {osservazione}{Osservazione}
\newtheorem{proposizione}{Proposizione}
\newtheorem{esempio}{Esempio}
\newtheorem{corollario}{Corollario}
\newtheorem{teorema}{Teorema}